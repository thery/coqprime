

\section{Safe computation and prime numbers}

Recent formal developments such as ~\cite{4color,kepler} have shown all the benefits
one can get from having a formal system where both proving and computing are
possible. In the {\sc Coq} proof assistant~\cite{Coq}, computation is
provided by the logic. {\sc Coq} is based on the Calculus of 
Inductive Construction, so the evaluation mechanism is given for free
by the beta reduction rule. 

A direct application of the primitive status of computation is 
the so-called two level approach~\cite{boutin}. To illustrate it, 
let us consider the problem of proving the primality of some natural 
numbers.
Suppose that we have defined a predicate {\it prime}.
For example, a number is prime if it has exactly two divisors: one and
itself. How do we now prove that 17 is prime? The standard approach is to
directly build a proof object using tactics. Of course, this task can be 
automated writing an ad-hoc tactic. Still, behind the scene, the system 
will have to build a proof object and the larger the number to be proved
prime is, the larger the proof term will be.
The two level approach proposes an alternative
strategy in two steps. In the first step, one defines a function that expresses
the problem in term of pure computation. In our case, it amounts in writing a 
function {\tt test} from natural number to boolean such that the function
returns {\tt true} if the number is prime. For example, if the natural number
is $n$, the function can check that there is no divisor between 2 and $n-1$
by a simple iteration. In the second step, one proves that the function meets 
its specification
$$
\forall n, \textit{test}\,\, n = \texttt{true} \rightarrow \textit{prime}\,\, n
$$
Now to give a proof that 17 is prime, it is sufficient to prove that the function
{\tt test} applies to 17 returns {\tt true}. As the function {\it test} directly
evaluates inside {\sc Coq}, this last proof is simply the reflexivity of the equality.
Using the two level approach, we have just transfered the problem of building a 
large proof object into a conversion problem: showing that $test\,\, 17$ is convertible
to 17.  The size of the proof object is then independent of the number to be proved
prime. Recent progress in the evaluation mechanism~\cite{GreLer} has also made this
approach attractive from the point of view of efficiency.

The motivation of what is presented in this paper comes from an earlier work 
presented in~\cite{GreTheWer}. In this work, a more elaborated way of applying 
the two level approach is proposed to prove primality. It is based on the
notion of prime certificate and more precisely of {\it Pocklington certificate}.
A prime certificate is an object that witnesses the primality of a number.
The Pocklington certificates we have been using are justified by the following
theorem given in~\cite{lehmer}:
\begin{theorem}
Given a number $n$, a witness $a$ and some pairs $(p_1,\alpha_1),\dots,(p_k,\alpha_k)$
 where all the $p_i$ are prime numbers,
 let
 \begin{itemize}
\item[]$F_1 = p_1^{\alpha_1}\dots p_k^{\alpha_k}$
\item[]$R_1 = (n - 1) / F_1$
\item[]$ s = R_1 / (2F_1)$
\item[] $r = R_1 \mod\ (2F_1)$
 \end{itemize}
 it is sufficient for $n$ to be prime that the following conditions hold:
\begin{eqnarray}
F_1 \,\,\hbox{is even},\,\,
R_1 \,\, \hbox{is odd}, \,\,\hbox{and}\,\,
F_1R_1  &=&  n -1\\
(F_1 + 1) (2F_1^2 + (r - 1) F_1 + 1) & >& n\\
a^{n-1} &=& 1 (\mod\ n)\\
\forall i\in\{1,\dots,k\}~\gcd(a^{\frac{n-1}{p_i}}-1,n)&=&1\\
r^2 - 8 s\,\,\mbox{is not a square}\,\,\hbox{or}\,\,s &=& 0
\end{eqnarray}
\end{theorem}
For a prime number $n$, the list $[a, p_1, \alpha_1, p_2, \alpha_2, \dots, p_k, \alpha_k]$
represents its Pocklington certificate.
Even if generating a certificate for a given $n$ can be cpu-intensive, verifying
the conditions 1-5 is an order of magnitude simpler. In fact, only
this last verification that is crucial for asserting the primality (so requires
safe computation) is done inside {\sc Coq}.
The generation of the certificate is delegated to an external tool.
This is a direct application of the skeptic approach described in~\cite{BarBar,HarThe}.
                 
The idea of using Pocklington certificate was not new and originally presented 
in~\cite{Caprotti_Oostdijk:01pockjsc}.
What was new was to apply the two level approach for verifying the
certificates. The improvement in term of size of the
proof object and in term of time is illustrated on some examples in
Figure~\ref{fig:TimeComp}.
\begin{figure}
\begin{center}
\begin{tabular}{|l|r|r|r|r|r|r|}
\cline{3-6}
\multicolumn{2}{c}{} & \multicolumn{2}{|c|}{size} &
                                               \multicolumn{2}{c|}{time} \\
\hline
~prime     ~     & \multicolumn{1}{c|}{~digits~ } & ~standard~  &
  \multicolumn{1}{c|}{ ~two level~ } &
  \multicolumn{1}{c|}{ ~standard~ } &
  \multicolumn{1}{c|}{ ~two level~ } \\
\hline
~1234567891       ~   & 10~ &  94K~ &  ~0.453K~ & 3.98s~  & 0.50s~  \\
~74747474747474747~   & 17~ & 145K~ &  0.502K~ &   9.87s~ & 0.56s~ \\
~1111111111111111111~ & 19~ & 223K~ &  0.664K~ & 17.41s~  & 0.66s~   \\
~$(2^{148}+1)/17$ ~   & 44~ & 1.2M~ & 0.798K~ & ~350.63s~  & 2.77s~   \\
~$P_{150}$   ~        &150~ &  \_~  & 1.902K~   &  \_~   & 75.62s~  \\
\hline
\end{tabular}
\end{center}
\caption{Some verifications of certificates with the standard and two level approaches}
\label{fig:TimeComp}
\end{figure} 
There was still room for improvement. When dealing
with relatively large numbers (> 1000 digits?) the verification was getting really
time consuming. This was particularly true when applying the very simple Lehmer test
for proving the primality of Mersenne numbers, i.e. numbers that can be written as $2 ^ p -1$.
\begin{theorem}\label{lucas}
Let ($S_n$) be recursively defined by $S_0= 4$ and $S_{n+1} = S_n^2 - 2$,
for $p > 2$, $2^p-1$ is prime if and only if $(2^p -1) | S_{p-2}$.
\end{theorem}
The bottle neck was clearly the default arithmetic provided by {\sc Coq}. If it was 
reasonable for small numbers, dealing with larger number was problematic.
The purpose of this paper is to propose a more adequate arithmetic for safe computations
with arbitrary large numbers within {\sc Coq}. If we consider the type of computations
we need, the most cpu-intensive verifications only require modular arithmetic.  
Condition 3 is directly a modulo. Condition 4 can be rephrased as
$$
\gcd(\prod_{i=1}^{i=k}a^{\frac{n-1}{p_i}} \mod\, n -1,n) = 1
$$
Finally an equivalent version of Theorem \ref{lucas} is
\begin{theorem}
Let $M_p$ be $2^p-1$ and let  ($S_p$) be recursively defined by $S_0= 4\, \mod\, M_p$ and $S_{n+1} = S_n^2 - 2\, \mod\, M_p$,
for $p > 2$, $M_p$ is prime if and only if $S_{p-2} = 0\, \mod\, M_p $.
\end{theorem}
 So we concentrate our effort on providing a library for modular arithmetic. The key idea of our
 library is to implement a representation of numbers that accommodates the divide and
 conquer strategy to speed-up computation. The paper is organised as follows. 
 In Section~\ref{Z}, we present the default arithmetic of {\sc Coq}.
 Our representation of numbers is described in Section~\ref{Tree}. In Section~\ref{Op}, we 
 give an overview of the functions that have been implemented and proved
 correct in our library. Section 5 presents some tests that validate our approach.
 In Section 6, we detail some further experiments that we have been carrying out.