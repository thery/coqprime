\section{Conclusions}


The main contribution of our work is to present a certified library for performing
modular arithmetic. Individual arithmetic functions have already been proved correct,
see for example~\cite{BerMagZim02}. To our knowledge, it is the first time verification
has been applied to a complete library  with non trivial algorithms successfully.
Our motivation was to be able to certify  larger prime numbers. Figures given in Section~\ref{bench} 
prove that this goal has been reached: we are now capable of manipulating numbers with more than 13000 digits.
These tests also show that the library with a native base arithmetic is
reasonably efficient. We hope it will motivate people to integrate machine word arithmetic inside {\sc Coq}.

Expressing the arithmetic in the logic has a price: no side effect is possible, 
also numbers are allocated progressively not in one block.
A natural continuation of our work would be to prove the correctness of a library with side effect.
This would require a much more intensive verification work since inplace computing
is known to be much harder to verify.
Note that directly integrating an existing library inside the prover with no verification
would go against the philosophy of {\sc Coq} to keep its trusted computing base as small
as possible.

From the methodological point of view, the most interesting aspect of this work
has been the use of the meta-proving technique to generate our base arithmetic. This has proved
to be a very powerful technique. We have used it in an ad-hoc way: files are generated concatenating
strings. Developing a more adequate support for meta-proving inside the prover seems a
very promising future work. Note that meta-proving could also be a solution to get more flexibility
in the proof system. Slightly changing our representation, adding for example to  the {\tt w2} type not only {\tt WO}
but also {\tt W1} and {\tt W-1}, would have a devastating effect on our definitions and proofs.
Meta-proving could be a solution for having a formal development for a family of data-structures rather than
just a single one.

Finally, on December 2005, a new prime Mersenne number has been discovered: $2 ^{30402457} - 1$.
It took 5 days to  perform its Lucas-Lehmer test on a super computer. 
The program uses a very intriguing algorithm to perform 
multiplication~\cite{crandall}. 
Proving the correctness of
such an algorithm seems a very challenging task. 
