\documentclass{llncs}
\usepackage{url}
\usepackage[latin1]{inputenc}
\usepackage[T1]{fontenc}
\usepackage{amsfonts}
\usepackage{latexsym}
\usepackage{color}  %
\usepackage{pst-node}
\usepackage{ae,aecompl,amsbsy,amssymb}
\title{A purely functional library for\\ modular arithmetic  and its application to\\ certifying large prime numbers}
\author{Benjamin Gr�goire \and Laurent Th�ry}
\institute{INRIA Sophia-Antipolis, France
\texttt{[Benjamin.Gregoire|Laurent.Thery]@sophia.inria.fr}}
\newcommand{\mod}{\mbox{mod}}
\newcommand{\WW}[2]{\texttt{WW}\,\,#1\,\,#2\,}
\newcommand{\wB}{{\texttt{wB}}}
\newcommand{\CI}[1]{\texttt{C1}\,\,#1}
\newcommand{\CO}[1]{\texttt{C0}\,\,#1}
\newcommand{\ICP}[1]{[\raisebox{1.5pt}{\tiny +}|#1|]}
\newcommand{\ICM}[1]{[\raisebox{1.5pt}{\tiny --}|#1|]}
\begin{document}

\maketitle



\begin{abstract}
Computing efficiently with numbers can be crucial for some theorem proving applications.
In this paper, we present a library of modular arithmetic that has been developed within
the {\sc Coq} proof assistant. The library proposes the usual operations that have all been
proved correct. The library is purely functional but can also be used on top of some native
modular arithmetic. 
With this library, we have been capable of certifying the primality of numbers with
more than 20000 digits.

\end{abstract}



\section{Safe computation and prime numbers}

Recent formal developments such as ~\cite{4color,kepler} have shown all the benefits
one can get from having a formal system where both proving and computing are
possible. In the {\sc Coq} proof assistant~\cite{Coq}, computation is
provided by the logic. {\sc Coq} is based on the Calculus of 
Inductive Construction, so the evaluation mechanism is given for free
by the beta reduction rule. 

A direct application of the primitive status of computation is 
the so-called two-level approach~\cite{boutin}. To illustrate it, 
let us consider the problem of proving the primality of some natural 
numbers.
Suppose that we have defined a predicate {\it prime}: a number is prime 
if it has exactly two divisors, one and itself. 
How do we now prove that 17 is prime? The standard approach is to
directly build a proof object using tactics. Of course, this task can be 
automated writing an ad-hoc tactic. Still, behind the scene, the system 
will have to build a proof object and the larger the number to be proved
prime is, the larger the proof term will be.
The two-level approach proposes an alternative
strategy in two steps. In the first step, one defines a function that expresses
the problem in term of pure computation. It can be seen as a semi-decision procedure.
In our case, it amounts in writing a 
function {\tt test} from natural number to boolean such that the function
returns {\tt true} if the number is prime. For example, if the natural number
is $n$, the function can check that there is no divisor between 2 and $n-1$
by a simple iteration. In the second step, one proves that the function meets 
its specification
$$
\forall n, \textit{test}\,\, n = \texttt{true} \rightarrow \textit{prime}\,\, n
$$
So our decision procedure is correct.
Now to give a proof that 17 is prime, it is sufficient to prove that the function
{\tt test} applies to 17 returns {\tt true}. As the function {\it test} directly
evaluates inside {\sc Coq}, this last proof is simply the reflexivity of equality.
Using the two-level approach, we have just transfered the problem of building a 
large proof object into a conversion problem: showing that $test\,\, 17$ is convertible
to 17.  The size of the proof object is then independent of the number to be proved
prime. Recent progress in the evaluation mechanism~\cite{GreLer} has also made this
approach attractive from the point of view of efficiency.

In~\cite{GreTheWer} we have presented a more elaborated way of applying 
the two-level approach for proving primality. It is based on the
notion of prime certificate and more precisely of {\it Pocklington certificate}.
A prime certificate is an object that witnesses the primality of a number.
The Pocklington certificates we have been using are justified by the following
theorem given in~\cite{lehmer}:
\begin{theorem}\label{lehmer}
Given a number $n$, a witness $a$ and some pairs $(p_1,\alpha_1),\dots,(p_k,\alpha_k)$
 where all the $p_i$ are prime numbers,
 let
 \begin{itemize}
\item[]$F_1 = p_1^{\alpha_1}\dots p_k^{\alpha_k}$
\item[]$R_1 = (n - 1) / F_1$
\item[]$ s = R_1 / (2F_1)$
\item[] $r = R_1 \mod\ (2F_1)$
 \end{itemize}
 it is sufficient for $n$ to be prime that the following conditions hold:
\begin{eqnarray}
F_1 \,\,\hbox{is even},\,\,
R_1 \,\, \hbox{is odd}, \,\,\hbox{and}\,\,
F_1R_1  &=&  n -1\\
(F_1 + 1) (2F_1^2 + (r - 1) F_1 + 1) & >& n\\
a^{n-1} &=& 1 (\mod\ n)\\
\forall i\in\{1,\dots,k\}~\gcd(a^{\frac{n-1}{p_i}}-1,n)&=&1\\
r^2 - 8 s\,\,\mbox{is not a square}\,\,\hbox{or}\,\,s &=& 0
\end{eqnarray}
\end{theorem}
For a prime number $n$, the list $[a, p_1, \alpha_1, p_2, \alpha_2, \dots, p_k, \alpha_k]$
represents its Pocklington certificate.
Even if generating a certificate for a given $n$ can be cpu-intensive, verifying
the conditions 1-5 is an order of magnitude simpler. In fact, only
the verification of the conditions 1-5 is crucial for asserting the primality. It requires
safe computation and is done inside {\sc Coq}.
The generation of the certificate is delegated to an external tool.
This is a direct application of the skeptic approach described in~\cite{BarBar,HarThe}.
                 
With respect to the standard approach for the same problem~\cite{Caprotti_Oostdijk:01pockjsc}, the two-level
approach gives a huge improvement in term of size of the
proof object and in term of time.  Figure~\ref{fig:TimeComp} illustrates this
on some small examples.
\begin{figure}
\begin{center}
\begin{tabular}{|l|r|r|r|r|r|r|}
\cline{3-6}
\multicolumn{2}{c}{} & \multicolumn{2}{|c|}{size} &
                                               \multicolumn{2}{c|}{time} \\
\hline
~prime     ~     & \multicolumn{1}{c|}{~digits~ } & ~standard~  &
  \multicolumn{1}{c|}{ ~two-level~ } &
  \multicolumn{1}{c|}{ ~standard~ } &
  \multicolumn{1}{c|}{ ~two-level~ } \\
\hline
~1234567891       ~   & 10~ &  94K~ &  ~0.453K~ & 3.98s~  & 0.50s~  \\
~74747474747474747~   & 17~ & 145K~ &  0.502K~ &   9.87s~ & 0.56s~ \\
~1111111111111111111~ & 19~ & 223K~ &  0.664K~ & 17.41s~  & 0.66s~   \\
~$(2^{148}+1)/17$ ~   & 44~ & 1.2M~ & 0.798K~ & ~350.63s~  & 2.77s~   \\
~$P_{150}$   ~        &150~ &  \_~  & 1.902K~   &  \_~   & 75.62s~  \\
\hline
\end{tabular}
\end{center}
\caption{Some verifications of certificates with the standard and two-level approaches}
\label{fig:TimeComp}
\end{figure} 
Applying the two-level approach to larger numbers (> 1000 digits) made us realize
the algorithmic limitation of the arithmetic provided by {\sc Coq}.
This was particularly true when applying the Lucas-Lehmer test
for proving the primality of Mersenne numbers, i.e. numbers that can be written as $2 ^ p -1$.
\begin{theorem}\label{lucas}
Let ($S_n$) be recursively defined by $S_0= 4$ and $S_{n+1} = S_n^2 - 2$,
for $p > 2$, $2^p-1$ is prime if and only if $(2^p -1) | S_{p-2}$.
\end{theorem}
The largest Mersenne number we could certify was $2^{4423} - 1$. 
The purpose of this paper is to propose a more adequate arithmetic for safe 
computations with arbitrary large numbers. It has been implemented and its correctness
has been proved within the {\sc Coq} system. The arithmetic that is implemented is a modular
one. This is not a real limitation since most computations with large numbers
can be rephrased in term of modular arithmetic. This is true in particular for the examples we are
interested in. More precisely, Condition 4 of Theorem \ref{lehmer} can be expressed with modulo as
$$
\gcd(\prod_{i=1}^{i=k}a^{\frac{n-1}{p_i}} \mod\, n -1,n) = 1
$$
and Theorem \ref{lucas} can be rephrased as
\begin{theorem}
Let $M_p$ be $2^p-1$ and let  ($S_n$) be recursively defined by $S_0= 4\, \mod\, M_p$ and $S_{n+1} = S_n^2 - 2\, \mod\, M_p$,
for $p > 2$, $M_p$ is prime if and only if $S_{p-2} = 0\, \mod\, M_p $.
\end{theorem}
The key idea of our library is to implement a representation of numbers that accommodates the divide and
conquer strategy to speed up computation. The paper is organised as follows. 
In Section~\ref{ZTree}, we confront the standard arithmetic of {\sc Coq} with
our representation of numbers. In Section~\ref{Op}, we give an overview of the library.  
In Section~\ref{word}, we detail two possible instantiations of our library.
Finally, Section~\ref{bench} presents some tests that validate our approach.
 
\section{Linear versus tree representation of numbers\label{ZTree}}

In the standard library of {\sc Coq}, strictly positive numbers are represented as linear structures, low bits first.

\begin{verbatim}
Inductive positive : Set :=
  | xI : positive -> positive 
  | xO : positive -> positive 
  | xH : positive.
\end{verbatim}
 {\tt xH} is 1, {\tt (xO p)} is two times the value of {\tt p}
 and {\tt (xI p)} is two times plus one the value of {\tt p}.
For example, 17 and 18 are represented as {\tt xI (xO (xO (xO (xH))))} and
{\tt xO (xI (xO (xO (xH))))} respectively. The choice of the representation
has some direct impact on the way operations are implemented. To illustrate this on an example,
let us consider the comparison function {\tt Pcmp}. It takes two positive numbers
and returns a comparison value
\begin{verbatim}
Inductive comparison: Set := Eq | Lt | Gt.
\end{verbatim}
As numbers are represented low bits first, to compare two numbers 
one needs to walk down both numbers keeping track of what the current status
of the comparison is. This is what the auxiliary function {\tt Pcompare} does. The main 
function {\tt Pcomp} starts the computation with the initial status being equality.
\begin{verbatim}
Fixpoint Pcompare (x y: positive) (r: comparison): comparison :=
  match x, y with
  |    xH,    xH => r
  |    xH,    _  => Lt
  |    _ ,    xH => Gt
  | xI x', xI y' => Pcompare x' y' r
  | xO x', xO y' => Pcompare x' y' r
  | xI x', xO y' => Pcompare x' y' Gt
  | xO x', xI y' => Pcompare x' y' Lt
  end.
  
Definition Pcmp x y := Pcompare x y Eq.
\end{verbatim}
This is clearly not optimal but is the best one can do with this 
representation: recursive calls only skip a single bit. 
Efficient algorithms for large numbers, like Karatsuba multiplication~\cite{Karat}, 
use a divide and conquer strategy. They require to be able to split numbers in parts efficiently. 
This motivates our representation based on a tree-like structure. 
Given an arbitrary one-word set {\tt w}, we define the two-word  set {\tt w2 w} as follows 
\begin{verbatim}
Inductive w2 (w: Set): Set :=  WW : w -> w -> w2 w.
\end{verbatim}
For example, {\tt (WW true false)} is of type {\tt (w2 bool)}.
We choose in an arbitrary way  that high bits are the first argument of {\tt WW}, low bits
the second one. Now we use a recursive type definition and define
the type of numbers of height {\tt n} as
\begin{verbatim}
Fixpoint word (w: Set) (n:nat): Set :=
 match n with
 |   O => w
 | S n => w2 (word w n)
 end.
\end{verbatim}
An object of type {\tt (word w n)} is a complete binary tree that
contains $2^\texttt{n}$ objects of type {\tt w}. Given a number,
one has to choose an appropriate height to represent it exactly.
For example, taking the usual booleans for base words, a minimum
height of 2 is necessary to represent the number 13. With this height, 
numbers have type {\tt (word bool 2)} and 
{\tt (WW (WW true true) (WW false true))} denotes the number 13.

Arithmetic operations are not going to be defined on the type {\tt word} directly.
We use a technique similar to the one in~\cite{GreMa}. A functor is first defined that 
allows to build a two-word modular arithmetic on top of a single-word one.
The functor is then applied iteratively to get the final implementation.
In the following, $x,\ y$ are used to denote single-word variables and $\textit{xx},\ \textit{yy}$ 
to denote two-word variables. 
When defining a new function $f$, we just need to explain how to compute the 
result on two-word values knowing how to compute it on one-word values.
We use the notation $w\_f$ for the single-word version of $f$ and 
$\textit{ww}\_f$ for the two-word version.
For example, let us go back to our comparison function {\tt Pcompare}
and try to define it on our trees. We first suppose the existence
of the comparison on single words
\begin{verbatim}
Variable w_compare: w -> w -> comparison -> comparison.
\end{verbatim}
and then define the function for two-word values
\begin{verbatim}
Definition ww_compare (xx yy: w2 w) (r: comparison) :=
  match xx, yy with
    WW xH xL, WW yH yL => w_compare xH yH (w_compare xL yL r) 
  end.
\end{verbatim}
This is not the function that is in our library. Instead, we can take 
advantage of the tree-like structure and compare high bits first.  
\begin{verbatim}
Variable w_cmp: w -> w -> comparison.

Definition ww_cmp (xx yy: w2 w) :=
  match xx, yy with
    WW xH xL, WW yH yL => 
     match w_cmp xH yH with Eq => w_cmp xL yL | cmp => cmp end
  end. 
\end{verbatim}
The key property of our representation is that splitting number in two
is for free. The next section details why this property is crucial to implement efficient
algorithms for functions like multiplication, division and square root.
Note that, in term of memory allocation, having a tree structure
does not produce any overhead. In a functional setting, building a binary 
tree structure or building the equivalent linear list of words requires the same number of cells.

One main drawback of our representation is that we manipulate only complete 
binary trees. 
So, even if we choose carefully the appropriate height, half of the words 
could be unnecessary to compute the final result. 
To soften this problem, we have extended the definition of {\tt w2} 
to include an empty word {\tt W0}. 
\begin{verbatim}
Inductive w2 (w: Set): Set :=  
|  W0: w2
|  WW: w -> w -> w2.
\end{verbatim}
For example, the number 13 can be represented at height 3 as
\begin{verbatim}
 WW W0 (WW (WW true true) (WW false true))
\end{verbatim}
With this extension, we lose uniqueness of representation. Still, there is a notion
of canonicity, {\tt W0} should always be preferred to a sub-tree full of zeros. Note that, in
our development, all functions have been carefully written in order to preserve canonicity, but
canonicity  is not part of their specification since it is not necessary to ensure safe computations.
Using {\tt w\_0} to represent the one-word zero, 
the final version of the comparison function is then
\begin{verbatim}
Definition ww_cmp (xx yy: w2 w) :=
  match xx, yy with
  | W0, W0 => Eq
  | W0, WW yH yL =>
     match w_cmp w_0 yH  with Eq => w_cmp w_0 yL  | _   => Lt  end
  | WW xh xl, W0 =>
     match w_cmp xH  w_0 with Eq => w_cmp xL  w_0 | _   => Gt  end
  | WW xH xL, WW yH yL =>
     match w_cmp xH  yH  with Eq => w_cmp xL  yL  | cmp => cmp end
  end.
\end{verbatim}



\section{The certified library \label{Op}}
Our library includes the usual functions:
comparison, successor, predecessor, opposite, addition, subtraction,
multiplication, square, euclidean division, integer square root, gcd, 
power and modulo.
It is a modular library : it means that we only manipulate tree (or word)
of the same height (of the same size). 
For addition and subtraction, we also provide an exact version
that returns a word and a carry.
For multiplication, we also provide an exact version
returning two words. 

Since we want to use our library in the context of a two-level approach,
we must carefully choose the algorithms we implement.
Furthermore, semi-decision procedures must also be certified, 
so every function of our library must come along with its proof 
of correctness. 

Specifications are expressed using predicates over integers. For this, 
we use two interpretation functions
{$ [|\ |]$} and {$[[\ ]]$}.
Given a single-word element $x$, its corresponding integer value
is $[|x|]$. Given a two-word element $xx$, its corresponding
integer value is $[[xx]]$. The base of the arithmetic, i.e.
one plus the maximum value that fits in a single-word, is $\wB$. 
We write {\tt w\_0} (resp. {\tt w\_1}) for the word with 
corresponding integer value 0 (resp. 1).
From these definitions, the following statement holds
$$\forall x\, y,\, [[\WW x  y]] = [|x|] * \wB + [|y|]$$
Once a function is defined, its correctness has to be proved.
For example for the comparison defined in the previous section,
one needs to prove that if the function {\tt w\_cmp} meets its specification
$$\begin{array}{l}
\forall x\, y, \,\, \textit{match {\tt w\_cmp} x y with} \\
\qquad\,\,\,\,
 |\,\, \texttt{Eq}\,\, \rightarrow [|x|] = [|y|]\,\, | \,\,\texttt{Lt}\,\, \rightarrow [|x|] < [|y|] \,\,
| \,\,\texttt{Gt}\,\, \rightarrow [|x|] > [|y|]\\
\qquad\,\,\,\, \textit{end}
\end{array}
$$
so does the function {\tt ww\_cmp}
$$\begin{array}{l}
\forall xx\, yy, \,\, \textit{match {\tt ww\_cmp} xx yy with} \\
\qquad\,\,\,\,
 |\,\, \texttt{Eq}\,\, \rightarrow [[xx]] = [[yy]]\,\, | \,\,\texttt{Lt}\,\, \rightarrow [[xx]] < [[yy]] \,\,
| \,\,\texttt{Gt}\,\, \rightarrow [[xx]] > [[yy]]\\
\qquad\,\,\,\, \textit{end}
\end{array}
$$

\subsection{Words and carries}

Carries are important for operations like addition and subtraction.
In our functional setting, carries encapsulate words
\begin{verbatim}
Inductive carry (w: Set): Set :=
  | C0: w -> carry
  | C1: w -> carry.
\end{verbatim}
Two interpretation functions are associated with carries.
One interprets the carry positively: $\ICP{\CI x} = \wB + [|x|]$ and
$\ICP{\CO x} = [|x|]$. 
The other one interprets it negatively (i.e. a borrow):
$\ICM{\CI x} = [|x|] - \wB$ and $\ICM{\CO x} = [|x|]$.
To illustrate how carries are manipulated, let us consider the successor function.
In our library, it is represented by two functions
\begin{verbatim}
w_succ: w -> w
w_succ_c: w -> carry w
\end{verbatim}
The first function represents the modular version, the second one the exact 
version. With these two functions, it is possible to define the version
for two-word elements. For example, the definition for the modular version is
\begin{verbatim}
 Definition ww_succ xx :=
  match xx with
  | W0 => WW w_0 w_1
  | WW xH xL =>
    match w_succ_c xL with
    | C0 l => WW xH l
    | C1 l => WW (w_succ xH) w_0
    end
  end.
\end{verbatim}
Note that unlike what happens in imperative languages, returning
a carry allocates a memory cell. So in our implementation
we avoid as much as possible to create them. When we know in advance 
that the result always returns (respectively does not return) a carry, we can call the modular function instead. 
An example of such a situation is a naive implementation of the exact function that adds two to a 
one-word element by calling twice the successor function.
\begin{verbatim}
 Definition w_add2 x :=
  match w_succ_c x with
  | C0 y => w_succ_c y
  | C1 y => C1 (w_succ y)
  end.
\end{verbatim}
In the case where the first increment has created a carry, we are sure that the second
increment cannot raise any carry so we can directly call the {\tt w\_succ} function.
Also, we use a combination of partial evaluation and continuation passing style to get
shorter definitions. This has proved to ease considerably the proving phase without
changing the efficiency of functions.

\subsection{Shifting bits}

If most of the operations work at word level, some functions like the
shifting operation require to work at a lower level: the bit level.
Surprisingly, all the operations we had to perform at bit level can be built
using a single function
\begin{verbatim}
w_add_mul_div : positive -> w -> w -> w
\end{verbatim}
Evaluating  {\tt (w\_add\_mul\_div p x y)} returns a new word that
is composed for its last {\tt p} bits by the first bits of {\tt y}
and for the remaining bits by the last bits of {\tt x}.
Its specification is 
$$\begin{array}{l}
\forall p\, x\, y, \,\, 2 ^ p < \wB  \Rightarrow \\
\hskip20pt [|\textit{\tt{w\_add\_mul\_div}}\, p\, x\, y|] = ([|x|] * 2 ^ p + ([|y|] * 2 ^ p) / \wB) \,\mod\, \wB
\end{array}
$$
Two degenerated versions of this function are of direct interest. Calling it
%{\tt w\_add\_mul\_div} 
with a zero word as second argument implements
the shift left. Calling it with a zero word as third argument implements
the shift right.

\subsection{Divide and conquer algorithms}

\subsubsection{Karatsuba multiplication}
Speeding up the multiplication was the main motivation of our 
tree representation for numbers. The multiplication is represented in our library
by the function
\begin{verbatim}
w_mul_c: w -> w -> w2 w
\end{verbatim}
and its specification is
$$\forall x\, y,\, [[\texttt{w\_mul\_c}\, x\, y]] = [|x|] * [|y|]$$
The naive implementation on two-word elements follows the simple equation
$$\begin{array}{l}
[[\WW {x_h} {x_l}]] * [[\WW {y_h} {y_l}]] =\\
\hskip20pt [|{x_h}|] * [|{y_h}|] * \wB ^ 2 +  ([|{x_h}|] * [|{y_l}|] + [|{x_l}|] * [|{y_h}|]) * \wB + [|{x_l}|] * [|{y_l}|]
\end{array}
$$
Thus, performing a multiplication requires four submultiplications.
Karatsuba multiplication~\cite{Karat} saves one of these submultiplications
$$\begin{array}{l}
[[\WW {x_h} {x_l}]] * [[\WW {y_h} {y_l}]] = \\
\hskip20pt \textit{let}\,\, {h}\, = [|{x_h}|] * [|{y_h}|]\,\, \textit{in}\\
\hskip20pt \textit{let}\,\,\,  {l}\, = [|{x_l}|] * [|{y_l}|]\,\, \textit{in}\\
\hskip20pt {h} * \wB^2 + (([|{x_h}|] + [|{x_l}|]) * ([|{y_h}|] + [|{y_l}|]) - {h} - {l}) * \wB + {l} \qquad\qquad
\end{array}
$$
If the above equation looks simple,  Karatsuba multiplication has been the most difficult function 
to implement and prove correct.
The difficulty is in the computation of the cross
product $(([|{x_h}|] + [|{x_l}|]) * ([|{y_h}|] + [|{y_l}|]) - {h} - {l})$. 
The inner product cannot be directly computed using the word multiplication: 
both additions $([|{x_h}|] + [|{x_l}|])$ and $([|{y_h}|] + [|{y_l}|])$ can have a carry, 
so their product fits in two words and a maximum carry of 3.
Also, we know that the final cross product is positive and fits in
two words and a maximum carry of 1. This is useful to optimize the code.
For example, 
if the inner product has no carry, we are sure that the subtractions 
by $h$ and $l$ cannot generate a borrow. Respectively, 
if the inner product has a carry of 3,  we are sure that each 
subtraction generates a borrow. In both cases, modular subtraction can 
be used instead of the exact one. 

Karatsuba multiplication is more efficient than the naive one
only when numbers are large enough. So our library
includes both implementations. They are used separately to define two
different functors. The functor with the naive multiplication is only used 
for trees of "small" height. 

\subsubsection{Recursive Division}
The general division algorithm that we have implemented is the usual schoolboy 
method that iterates the division of two words by one word. 
It is then crucial to perform this two-by-one division efficiently. 
The algorithm we have implemented is the one presented in~\cite{RecDiv}.
The idea is to use the recursive call on high bits to guess an approximation 
of the quotient and  then to perform an appropriate adjustment to get the exact quotient.

In our development, the two-by-one division takes
three words and returns a pair composed of the quotient and the remainder.
\begin{verbatim}
Variable w_div21: w -> w -> w -> w * w
\end{verbatim}
and its specification is 
$$\begin{array}{l}
\forall x_1\,\, x_2\,\, y,\, \textit{let}\,\, q,\, r = \texttt{w\_div21}\, x_1\,\, x_2\,\, y\,\, \textit{in}\,\, \\
\hskip10pt [|x_1|] < [|y|] \Rightarrow \wB / 2 \le [|y|] \Rightarrow  [[\WW {x_1} {x_2}]] = [|q|] * [|y|] + [|r|] \land 0 \le [|r|] < [|y|]
\end{array}
$$
The two conditions deserve some explanation.
The first one ensures that the quotient fits in one word.
The second one %requires the first bit of the divisor to be a one. It 
ensures that the recursive call computes an approximation 
of the quotient that is not too far from the correct value.

Before defining the function {\tt ww\_div21} for two-word elements,
we need to define the intermediate function  {\tt w\_div32}
that divides three one-word elements by two one-word elements.
Its specification is
$$\begin{array}{l}
\forall x_1\,\, x_2\,\, x_3\,\, y_1\,\, y_2, \,\, \textit{let}\,\, q,\, rr = 
\texttt{w\_div32}\,\, x_1\,\, x_2\,\, x_3\,\, y_1\,\, y_2\,\, \textit{in}\\
 \hskip20pt [[\WW {x_1} {x_2}]] < [[\WW {y_1} {y_2}]] \Rightarrow 
 \wB / 2 \le [|y_1|] \Rightarrow\\
 \hskip30pt [|x_1|] * \wB ^ 2 + [|x_2|] * \wB  + [|x_3|] =   [|q|] *  
 [[\WW {y_1} {y_2}]] + [[rr]] \,\,\land  \\
 \hskip30pt 0 \le [[rr]] < [[\WW {y_1} {y_2}]]
\end{array}
$$
The two conditions play the same roles than the ones for the specification of {\tt w\_div21}.
As the code is a bit intricate, we just explain here how the function
proceeds. It first calls {\tt w\_div21} to divide $x_1$ and $x_2$
by $y_1$. This gives a pair $q$ and $r$ such that
$$[|x_1|] * \wB + [|x_2|] = [|q|] * [|y_1|] + [|r|]$$
$q$ is considered as the approximation of the final quotient.
The condition $\wB / 2 \le [|y_1|]$ ensures that if this approximation
is not exact, it exceeds the real value of at most two units. So the quotient 
can only be $q$, $q - 1$ or $q - 2$. 
As we have 
$$[|x_1|] *  \wB ^ 2  + [|x_2|] * \wB + [|x_3|] = [|q|] * [[\WW {y_1} {y_2}]] + ([[\WW r {x_3}]] - [|q|] * [|y_2|])$$
we know in which situation we are by testing the sign of the candidate
remainder.
In our modular arithmetic, it amounts in checking if the subtraction of
$(\texttt{w\_mul\_c}\, q\,\, y_2)$ from $(\WW r {x_3}\!)$
produces or not a borrow.  
If it is positive or zero (no borrow), the quotient is $q$. 
If it is negative (a borrow), 
we have to consider $q - 1$ and add in consequence $(\WW {y_1} {y_2}\!)$ 
to the candidate remainder.
We test again the sign of this new candidate. 
If it is positive, the quotient is $q - 1$
otherwise it is $q - 2$.
Now, the definition of {\tt ww\_div21} is straightforward. 
Forgetting the {\tt W0}
constructor, we have
\begin{verbatim} 
 Definition ww_div21 xx1 xx2 yy :=
  match xx1, xx2, yy with
  ....  
  | WW x1H x1L, WW x2H x2L, WW yH yL =>
        let (qH, rr) := w_div32 x1H x1L x2H yH yL in
        match rr with
        | W0 => (WW qH w_0, WW w_0 x2L)
        | WW rH rL =>
          let (qL, s) := w_div32 rH rL x2L yH yL in
          (WW qH qL, s)
        end
  end.
\end{verbatim}
These two divisions can only be used if the divisor $y$ is greater or equal
than $\wB/2$.
This is not restrictive
because if $y$ is too small we can always find a $n$ such that 
$y* 2 ^n \ge \wB/2$. If we have $x * 2 ^ n  =  q  * (y * 2 ^ n) + r$ 
for some $x$ and $r$, then $r$ can be written as $r = 2^n * r'$ so
$x = q * y + r'$. So to perform the division of two numbers of the same size, 
we first shift divisor and dividend by $n$. 
The shifted dividend fits in two words and its high part is smaller
than the shifted divisor. Then, we use the division of two by one. 
The resulting quotient is correct, we just have to unshift the remainder.

\subsubsection{Recursive Square Root}

The algorithm for computing the square root is similar to the one for division.
It was first described in~\cite{RecSqrt} and has already been formalized in a 
theorem prover~\cite{BerMagZim02}. It requires the number to be split
in four. For this reason it is represented by the following function in our
library
\begin{verbatim}
w_sqrt2: w -> w -> w * carry w;
\end{verbatim}
The function returns the square root and the rest.
Its specification is
$$\begin{array}{l}
\forall x\, y,\,\, \textit{let}\,\, s,\, r = \texttt{w\_sqrt2}\, x\, y\,\, \textit{in}\\
\hskip10pt \wB / 4 \le [|x|] \Rightarrow [[\WW x y]] = {[|s|]} ^ 2 + \ICP{r} \land \ICP{r} \le 2 * [|s|]
\end{array}
$$
As for division, the input must be large enough so that the recursive call
that computes the approximation is not too far from the exact value. 

The definition of the square root needs a support function that implements a division
by two times a number
\begin{verbatim}
w_div2s: carry w -> w -> w -> carry w * carry w
\end{verbatim}
with its specification
$$\begin{array}{l}
\forall x_1\,\, x_2\,\, y, \,\, \textit{let}\,\, q,\, r = \texttt{w\_div2s}\,\, x_1\,\, x_2\,\, y\,\, \textit{in}\\
\hskip20pt \wB / 2 \le [|y|] \Rightarrow \ICP{x_1} \le 2 * [|y|] \Rightarrow\\
\hskip30pt \ICP{x_1} * \wB + [|x_2|] = \ICP{q} *  (2 * [|y|]) + \ICP{r} \,\, \land\,\, 0 \le \ICP{r} < 2 * [|y|]
\end{array}
$$
The idea of the algorithm is summarized by the following equation
$$\begin{array}{l}
\textit{let}\,\,\, {q_h} ,\, r\,\, = \texttt{w\_sqrt2}\,\, {x_h}\,\, {x_l}\,\, \textit{in}\\
\textit{let}\,\,\, {q_l} ,\, r_1 = \texttt{w\_div2s}\,\, r\,\, {y_h}\,\, {q_h}\,\, \textit{in}\\
{[[\WW {x_h} {x_l}]]} * \wB ^ 2 + [[\WW {y_h} {y_l}]] =
    [[\WW {q_h} {q_l}]] ^ 2 + \ICP{r_1} * \wB + [|{y_l}|] - {[|q_l|]} ^ 2 \\
\end{array}
$$
The element $(\WW {q_h} {q_l}\!)$ is a possible candidate
for the square root. Because of the condition on
the input, we are sure that the integer square root is either 
$(\WW {q_h} {q_l}\!)$ or
$(\WW {q_h} {q_l}\!) - 1$. It is the sign of $\ICP{r_1} * \wB + 
[|{y_l}|] - [|q_l|] ^ 2$ that indicates in which situation we are. 

\section{Implementing base word arithmetic \label{word}}

The final step to complete our library is to define the arithmetic for the base words. 
Once defined,  we get the modular arithmetic for the desired size by applying an appropriate
number of times our functors. In a classical implementation, these base words would machine words.
Unfortunately, machine words are not accessible from the {\sc Coq} language. 

\subsection{Defined modular arithmetic}

For the moment, the only way to have a modular arithmetic for base words inside {\sc Coq}
is to define base words as a datatype. For example, we have for two-bit words
\begin{verbatim}
Inductive word2 : Set := OO | OI | IO  | II.
\end{verbatim}
The functions are then defined by simple case analysis. For example,
the exact successor function is defined as
\begin{verbatim}
Definition word2_succ_c x :=
 match x with
 | OO => C0 OI
 | OI => C0 IO
 | IO => C0 II
 | II => C1 OO
 end.
\end{verbatim}
We also need to give the proofs that every function meets its specification.
These proofs are also done by case analysis. 

Rather than writing by hand functions and proofs, we have
written an {\sc Ocaml} program~\cite{ocaml} instead. This program takes the word
size as argument and generates the desired base arithmetics with all
its proofs. It is a nice application of meta-proving. Unfortunately, 
functions and their corresponding proofs grow quickly with the word size. 
For example, the addition for {\tt word8} is a pattern matching of 65536 cases. 
{\tt word8} is actually the largest size {\sc Coq} can handle. 

The main benefit of  this approach is to get an arithmetic library that is entirely expressed 
in the logic of {\sc Coq}. The library is portable: no extension of the {\sc Coq} kernel 
is needed. 

\subsection{Native modular arithmetic}

To test our library with some machine word arithmetic, we use the extraction
mechanism. It converts automatically {\sc Coq} functions into {\sc Ocaml} functions.
It is then possible to run the resulting program with the 31 bit native
{\sc Ocaml} arithmetic or a simulated 64 bit arithmetic. 
Not all the functions that we have implemented have their corresponding functions in the 
native modular arithmetic, so some native code had to be developed for these functions.
The formal verification of this code has not yet been completed.

\section{Evaluating the library \label{bench}}

A way of applying  the two-level approach for proving primality has been presented in~\cite{GreTheWer}.
It is based on the notion of prime certificate and more 
precisely of {\it Pocklington certificate}.
A prime certificate is an object that witnesses the primality of a number.
The Pocklington certificates we have been using are justified by the following
theorem given in~\cite{lehmer}:
\begin{theorem}\label{lehmer}
Given a number $n$, a witness $a$ and some pairs of natural numbers 
$(p_1,\alpha_1),\dots,(p_k,\alpha_k)$
 where all the $p_i$ are prime numbers,
 let
 \begin{itemize}
\item[]$F_1 = p_1^{\alpha_1}\dots p_k^{\alpha_k}$
\item[]$R_1 = (n - 1) / F_1$
\item[]$ s = R_1 / (2F_1)$
\item[] $r = R_1 \mod\ (2F_1)$
 \end{itemize}
 it is sufficient for $n$ to be prime that the following conditions hold:
\begin{eqnarray}
F_1 \,\,\hbox{is even},\,\,
R_1 \,\, \hbox{is odd}, \,\,\hbox{and}\,\,
F_1R_1  &=&  n -1\\
(F_1 + 1) (2F_1^2 + (r - 1) F_1 + 1) & >& n\\
a^{n-1} &=& 1 (\mod\ n)\\
\forall i\in\{1,\dots,k\}~\gcd(a^{\frac{n-1}{p_i}}-1,n)&=&1\\
r^2 - 8 s\,\,\mbox{is not a square}\,\,\hbox{or}\,\,s &=& 0
\end{eqnarray}
\end{theorem}
For a prime number $n$, the list 
$[a, p_1, \alpha_1, p_2, \alpha_2, \dots, p_k, \alpha_k]$
represents its Pocklington certificate.
Even if generating a certificate for a given $n$ can be cpu-intensive, 
verifying conditions (1)-(5) is an order of magnitude simpler (computing $a^m$ requires
a maximum of $2\textit{log}_2\, m$ multiplications). In fact, only
the verification of conditions (1)-(5) is crucial for asserting primality. 
This requires safe computation and is done inside {\sc Coq}.
The generation of the certificate is delegated to an external tool.
This is a direct application of the skeptic approach described 
in~\cite{BarBar,HarThe}. Note that this method of certifying prime numbers
is effective only if the prime number $n$ is such that  $n-1$ can be easily
partially factorised.
                 
With respect to the usual approach for the same 
problem~\cite{Caprotti_Oostdijk:01pockjsc}, the two-level
approach gives a significant improvement in terms of size of the
proof object and in terms of time.  
Figure~\ref{fig:TimeComp} illustrates this on some examples 
($P_{150}$ is a random prime number with 150 digits and the millennium prime is
a prime number with 2000 digits discovered by John B. Cosgrave).
However, due to the limitations of the linear representation of numbers in {\sc Coq},
even with the two-level approach, we were not capable of certifying large prime numbers 
(> 1000 digits) as illustrated by the millennium prime.
The same occurred when applying the Lucas-Lehmer test
for proving the primality of Mersenne numbers, 
i.e. numbers that can be written as $2 ^ p -1$.
\begin{theorem}\label{lucas}
Let ($S_n$) be recursively defined by $S_0= 4$ and $S_{n+1} = S_n^2 - 2$.
For $p > 2$, $2^p-1$ is prime if and only if $(2^p -1) | S_{p-2}$.
\end{theorem}
The largest Mersenne number we could certify was $2^{4423} - 1$ that has 1332 digits. 

The idea is then to use our new library based on a tree-like representation of numbers.
The complete library with the corresponding contribution for prime numbers 
is available at \url{http://gforge.inria.fr/projects/coqprime/}. 
It consists of 9000 lines of hand-written definitions and proofs. 
The automatically generated {\tt word8} arithmetic is much bigger,
95 Mb: 41 Mb are used to define functions and 54 Mb for the proofs. 
This is the largest ever contribution that has been verified by {\sc Coq}. 
With this new library, we have been capable of proving that the Mersenne number
$2^{44497} - 1$ was prime using {\sc Coq} with the Lucas-Lehmer test. 
As far as we know, it is the largest prime number  that has been certified 
by a theorem prover.

The certification with our library is faster even for  small numbers. 
This is illustrated in Figure~\ref{fig:TimeCompW} and the fifth and sixth  
columns of Figure~\ref{fig:Mersenne}. There is a maximum speed-up of 70. 
These benchmarks have been run on a Pentium 4 with 1 Gigabytes of RAM.

Comparing {\tt word8} with the 31-bit {\sc Ocaml} integer {\tt w31} shows 
all the benefit we could get from having machine words in {\sc Coq}. 
There is a maximum speed-up of 95 with respect to {\tt word8}. This
means a speed-up of 6650 with respect to the standard {\sc Coq} library.

The 64-bit {\sc Ocaml} integer {\tt w64} is a simulated arithmetic 
(our processor has only 32 bits).
This is why there is not such a gap between {\tt w31} and {\tt w64}. 
{\sc Big\_int}~\cite{bignum} is the  standard exact library for {\sc Ocaml}. 
It has a purely functional interface but 
is written in C. The comparison is not bad. For the last Mersenne, {\tt w64}
is only 4.6 times slower than {\sc Big\_int}. 
It is also very interesting that this gap is getting smaller as numbers get larger. 
On individual functions, random tests on addition give a ratio of 4 and on multiplication a ratio of 10.

We are still far away from getting the performance of the {\sc gmp}~\cite{GMP}
library. This library is written in C and uses in-place computation instead. 
This minimises considerably the number of memory allocations.
Unfortunately, in-place computation is not compatible with 
the logic of {\sc Coq}.

\begin{figure}
\begin{center}
\begin{tabular}{|l|r|r|r|r|r|r|}
\cline{3-6}
\multicolumn{2}{c}{} & \multicolumn{2}{|c|}{size} &
                                               \multicolumn{2}{c|}{time} \\
\hline
~prime     ~     & \multicolumn{1}{c|}{~digits~ } & ~standard~  &
  \multicolumn{1}{c|}{ ~two-level~ } &
  \multicolumn{1}{c|}{ ~standard~ } &
  \multicolumn{1}{c|}{ ~two-level~ } \\
\hline
~1234567891       ~   & 10~ &  94K~ &  ~0.453K~ & 3.98s~  & 0.50s~  \\
~74747474747474747~   & 17~ & 145K~ &  0.502K~ &   9.87s~ & 0.56s~ \\
~1111111111111111111~ & 19~ & 223K~ &  0.664K~ & 17.41s~  & 0.66s~   \\
~$(2^{148}+1)/17$ ~   & 44~ & 1.2M~ & 0.798K~ & ~350.63s~  & 2.77s~   \\
~$P_{150}$   ~        &150~ &  \_~  & 1.902K~   &  \_~   & 75.62s~  \\
~$\textit{millennium prime}$~   &2000 & \_~   & \_~      & \_~    & \_~ \\
\hline
\end{tabular}
\end{center}
\caption{Some verifications of certificates with the standard and two-level approaches}
\label{fig:TimeComp}
\end{figure} 
\begin{figure}
\begin{center}
\begin{tabular}{|l|r| r|r|}
\hline
 & ~digits~ & ~positive~ & ~word8~ \\
\hline
~1234567891       ~  & 10~  & 0.50s~  & 0.10s~  \\
~74747474747474747~  & 17~ & 0.56s~  & 0.12s~  \\
~1111111111111111111~ & 19~ & 0.66s~ & 0.20s~  \\
~$(2^{148}+1)/17$ ~   & 44~ & 2.77s~  & 0.36s~  \\
~$P_{150}$   ~       & 150~ & 75.62s~  & 8.44s~  \\
~$\textit{millennium prime}$~   &2000 & \_~   & 5320.05s~ \\
\hline
\end{tabular}
\end{center}
\caption{Some verifications of certificates with the standard and our {\sc Coq} arithmetics}
\label{fig:TimeCompW}
\end{figure} 
\begin{figure}
\begin{center}
\begin{tabular}{|r|r|r|r|r|r|r|r|r |}
\hline
\# & n & digits & year &  positive & word8 & w31 & w64 & Big\_int\\
\hline
12 &  127 &  39 & 1876 &  0.73s & 0.04s & 0.01s & 0.s & 0.s \\
13 &  521 & 157 & 1952 &  53.00s & 1.85s & 0.02s & 0.02s &  0.s\\
14 &  607 & 183 & 1952 &  84.00s & 2.78s & 0.03s & 0.03s &  0.s\\
15 & 1279 & 386 & 1952 &  827.00s & 20.21s& 0.25s & 0.16s &  0.02s\\
16 & 2203 & 664 & 1952 &  4421.00s & 89.1s & 1.1s & 0.8s &  0.08s\\
17 & 2281 & 687 & 1952 &  4964.00s & 97.59s & 1.21s & 0.82s &  0.09s\\
18 & 3217 & 969 & 1957 &  14680.00s & 237.65s & 2.85s & 2.14s &  0.22s\\
19 & 4253 & 1281 & 1961 &35198.00s & 494.09s& 6.4s & 4.58s &  0.6s\\
20 & 4423 & 1332 & 1961 &  39766.00s & 563.27s & 6.99s & 4.99s &  0.67s\\
21 & 9689 & 2917  & 1963 &   & 5304.08s & 56.1s & 39.98s &  5.89s\\	 
22 & 9941 & 2993  & 1963 &   & 5650.63s & 60.5s & 42.53s &  6.32s\\	 
23 & 11213 & 3376 & 1963 &    & 7607.00s & 80.56s & 57.47s &  11.25s\\ 
24 & 19937 & 6002  & 1971 &  & 34653.12s & 377.24s & 268.09s &  45.75s\\
25 & 21701 & 6533 & 1978 &  &43746.21s & 463.02s & 338.04s &  58.56s \\
26 & 23209 & 6987 & 1979  &  &51210.56s & 538.33s & 403.48s &  88.43s\\
27 & 44497 & 13395 & 1979  &  &282784.09s & 3282.23s & 2208.45s &  476.75s \\
\hline

\end{tabular}
\end{center}
\caption{Times to verify Mersenne numbers}
\label{fig:Mersenne}
\end{figure}

\section{Conclusions}

\begin{verbatim}
METTRE DANS LA CONCLUSION QUE CA NE S'APPLIQUE PAS SEULEMENT A LA PRIMALITE
\end{verbatim}

The main contribution of our work is to present a certified library for performing
modular arithmetic. Individual arithmetic functions have already been proved correct,
see for example~\cite{BerMagZim02}. To our knowledge, it is the first time verification
has been applied to a complete library  with non trivial algorithms successfully.
Our motivation was to be able to certify  larger prime numbers. Figures given in Section~\ref{bench} 
prove that this goal has been reached: we are now capable of manipulating numbers with more than 13000 digits.
These tests also show that the library with a native base arithmetic is
reasonably efficient. We hope it will motivate people to integrate machine word arithmetic inside {\sc Coq}.

Expressing the arithmetic in the logic has a prize: no side effect is possible, 
also numbers are allocated progressively not in one block.
A natural continuation of our work would be to prove the correctness of a library with side effect.
This would require a much more intensive verification work since inplace computing
is known to be much harder to verify.
Note that directly integrating an existing library inside the prover with no verification
would go against the philosophy of {\sc Coq} to keep its trusted computing base as small
as possible.

From the methodological point of view, the most interesting aspect of this work
has been the use of the meta-proving technique to generate our base arithmetic. This has proved
to be a very powerful technique. We have used it in an ad-hoc way: files are generated concatenating
strings. Developing a more adequate support for meta-proving inside the prover seems a
very promising future work. Note that meta-proving could also be a solution to get more flexibility
in the proof system. Slightly changing our representation, adding for example to  the {\tt w2} type not only {\tt WO}
but also {\tt W1} and {\tt W-1}, would have a devastating effect on our definitions and proofs.
Meta-proving could be a solution for having a formal development for a family of data-structures rather than
just a single one.

Finally, on December 2005, a new prime Mersenne number has been discovered: $2 ^{30402457} - 1$.
It took 5 days to  perform its Lucas-Lehmer test on a super computer. 
The program uses a very intriguing algorithm to perform 
multiplication~\cite{crandall}. 
Proving the correctness of
such an algorithm seems a very challenging task. 



\section*{Acknowledgments}
We would like to thank the anonymous referees for their careful reading of the paper and 
specially the referee who suggested a simplification to our implementation of Karatsuba multiplication.

\bibliographystyle{plain}
\bibliography{main}

\end{document}

