\documentclass{llncs}
\usepackage{url}
\usepackage[latin1]{inputenc}
\usepackage[T1]{fontenc}
\usepackage{amsfonts}
\usepackage{latexsym}
\usepackage{color}  %
\usepackage{pst-node}
\usepackage{ae,aecompl,amsbsy,amssymb}
\title{A purely functional library\\ for modular arithmetic  and its application\\ to certifying large prime numbers}
\author{Benjamin Gr�goire \and Laurent Th�ry}
\institute{INRIA Sophia-Antipolis, France
\texttt{[Benjamin.Gregoire|Laurent.Thery]@sophia.inria.fr}}
\newcommand{\mod}{\mbox{mod}}
\newcommand{\WW}[2]{\texttt{WW}\,\,#1\,\,#2\,}
\newcommand{\wB}{{\texttt{wB}}}
\newcommand{\CI}[1]{\texttt{C1}\,\,#1}
\newcommand{\CO}[1]{\texttt{C0}\,\,#1}
\newcommand{\ICP}[1]{[\raisebox{1.5pt}{\tiny +}|#1|]}
\newcommand{\ICM}[1]{[\raisebox{1.5pt}{\tiny --}|#1|]}
\begin{document}

\maketitle



\begin{abstract}
Computing efficiently with numbers can be crucial for some theorem proving applications.
In this paper, we present a library of modular arithmetic that has been developped within
the {\sc Coq} proof assitant.
With this library, we have been capable of certifying the primality of numbers with
more than 20000 digits.

\end{abstract}



\section{Safe computation and prime numbers}

Recent formal developments such as ~\cite{4color,kepler} have shown all the benefits
one can get from having a formal system where both proving and computing are
possible. In the {\sc Coq} proof assistant~\cite{Coq}, computation is
provided by the logic. {\sc Coq} is based on the Calculus of 
Inductive Construction, so the evaluation mechanism is given for free
by the beta reduction rule. 

A direct application of the primitive status of computation is 
the so-called two level approach~\cite{boutin}. To illustrate it, 
let us consider the problem of proving the primality of some natural 
numbers.
Suppose that we have defined a predicate {\it prime}: a number is prime 
if it has exactly two divisors: one and itself. 
How do we now prove that 17 is prime? The standard approach is to
directly build a proof object using tactics. Of course, this task can be 
automated writing an ad-hoc tactic. Still, behind the scene, the system 
will have to build a proof object and the larger the number to be proved
prime is, the larger the proof term will be.
The two level approach proposes an alternative
strategy in two steps. In the first step, one defines a function that expresses
the problem in term of pure computation. In our case, it amounts in writing a 
function {\tt test} from natural number to boolean such that the function
returns {\tt true} if the number is prime. For example, if the natural number
is $n$, the function can check that there is no divisor between 2 and $n-1$
by a simple iteration. In the second step, one proves that the function meets 
its specification
$$
\forall n, \textit{test}\,\, n = \texttt{true} \rightarrow \textit{prime}\,\, n
$$
Now to give a proof that 17 is prime, it is sufficient to prove that the function
{\tt test} applies to 17 returns {\tt true}. As the function {\it test} directly
evaluates inside {\sc Coq}, this last proof is simply the reflexivity of the equality.
Using the two level approach, we have just transfered the problem of building a 
large proof object into a conversion problem: showing that $test\,\, 17$ is convertible
to 17.  The size of the proof object is then independent of the number to be proved
prime. Recent progress in the evaluation mechanism~\cite{GreLer} has also made this
approach attractive from the point of view of efficiency.

In~\cite{GreTheWer} we have presented a more elaborated way of applying 
the two level approach for proving primality. It is based on the
notion of prime certificate and more precisely of {\it Pocklington certificate}.
A prime certificate is an object that witnesses the primality of a number.
The Pocklington certificates we have been using are justified by the following
theorem given in~\cite{lehmer}:
\begin{theorem}\label{lehmer}
Given a number $n$, a witness $a$ and some pairs $(p_1,\alpha_1),\dots,(p_k,\alpha_k)$
 where all the $p_i$ are prime numbers,
 let
 \begin{itemize}
\item[]$F_1 = p_1^{\alpha_1}\dots p_k^{\alpha_k}$
\item[]$R_1 = (n - 1) / F_1$
\item[]$ s = R_1 / (2F_1)$
\item[] $r = R_1 \mod\ (2F_1)$
 \end{itemize}
 it is sufficient for $n$ to be prime that the following conditions hold:
\begin{eqnarray}
F_1 \,\,\hbox{is even},\,\,
R_1 \,\, \hbox{is odd}, \,\,\hbox{and}\,\,
F_1R_1  &=&  n -1\\
(F_1 + 1) (2F_1^2 + (r - 1) F_1 + 1) & >& n\\
a^{n-1} &=& 1 (\mod\ n)\\
\forall i\in\{1,\dots,k\}~\gcd(a^{\frac{n-1}{p_i}}-1,n)&=&1\\
r^2 - 8 s\,\,\mbox{is not a square}\,\,\hbox{or}\,\,s &=& 0
\end{eqnarray}
\end{theorem}
For a prime number $n$, the list $[a, p_1, \alpha_1, p_2, \alpha_2, \dots, p_k, \alpha_k]$
represents its Pocklington certificate.
Even if generating a certificate for a given $n$ can be cpu-intensive, verifying
the conditions 1-5 is an order of magnitude simpler. In fact, only
this last verification that is crucial for asserting the primality (so requires
safe computation) is done inside {\sc Coq}.
The generation of the certificate is delegated to an external tool.
This is a direct application of the skeptic approach described in~\cite{BarBar,HarThe}.
                 
With respect to the standard approach for the same problem~\cite{Caprotti_Oostdijk:01pockjsc}, the two level
approach gives a huge improvement in term of size of the
proof object and in term of time.  Figure~\ref{fig:TimeComp} illustrates this
on some small examples.
\begin{figure}
\begin{center}
\begin{tabular}{|l|r|r|r|r|r|r|}
\cline{3-6}
\multicolumn{2}{c}{} & \multicolumn{2}{|c|}{size} &
                                               \multicolumn{2}{c|}{time} \\
\hline
~prime     ~     & \multicolumn{1}{c|}{~digits~ } & ~standard~  &
  \multicolumn{1}{c|}{ ~two level~ } &
  \multicolumn{1}{c|}{ ~standard~ } &
  \multicolumn{1}{c|}{ ~two level~ } \\
\hline
~1234567891       ~   & 10~ &  94K~ &  ~0.453K~ & 3.98s~  & 0.50s~  \\
~74747474747474747~   & 17~ & 145K~ &  0.502K~ &   9.87s~ & 0.56s~ \\
~1111111111111111111~ & 19~ & 223K~ &  0.664K~ & 17.41s~  & 0.66s~   \\
~$(2^{148}+1)/17$ ~   & 44~ & 1.2M~ & 0.798K~ & ~350.63s~  & 2.77s~   \\
~$P_{150}$   ~        &150~ &  \_~  & 1.902K~   &  \_~   & 75.62s~  \\
\hline
\end{tabular}
\end{center}
\caption{Some verifications of certificates with the standard and two level approaches}
\label{fig:TimeComp}
\end{figure} 
Applying the two level approach to larger numbers (> 1000 digits) made us realize
that the algorithmic limitation of the arithmetic provided by {\sc Coq}.
This was particularly true when applying the Lucas-Lehmer test
for proving the primality of Mersenne numbers, i.e. numbers that can be written as $2 ^ p -1$.
\begin{theorem}\label{lucas}
Let ($S_n$) be recursively defined by $S_0= 4$ and $S_{n+1} = S_n^2 - 2$,
for $p > 2$, $2^p-1$ is prime if and only if $(2^p -1) | S_{p-2}$.
\end{theorem}
The largest Mersenne number we could certify was $2^{4423} - 1$. 
The purpose of this paper is to propose a more adequate arithmetic for safe 
computations with arbitrary large numbers. It has been implemented and proved
correct within the {\sc Coq} system. The arithmetic that is implemented is a modular
one. This is not really a limitation since most computations with large numbers
can be rephrased in term of modular arithmetic. This is the case for the examples we are
interested in.
Condition 4 of Theorem \ref{lehmer} can also be expressed as
$$
\gcd(\prod_{i=1}^{i=k}a^{\frac{n-1}{p_i}} \mod\, n -1,n) = 1
$$
Theorem \ref{lucas} can be rephrased as
\begin{theorem}
Let $M_p$ be $2^p-1$ and let  ($S_p$) be recursively defined by $S_0= 4\, \mod\, M_p$ and $S_{n+1} = S_n^2 - 2\, \mod\, M_p$,
for $p > 2$, $M_p$ is prime if and only if $S_{p-2} = 0\, \mod\, M_p $.
\end{theorem}
The key idea of our library is to implement a representation of numbers that accommodates the divide and
conquer strategy to speed up computation. The paper is organised as follows. 
In Section~\ref{ZTree}, we confront the standard arithmetic of {\sc Coq} with
our representation of numbers. In Section~\ref{Op}, we give an overview of the functions 
that have been implemented and proved correct.  
In Section 5, we detail two alternative ways of using our library.
Finally, Section 6 presents some tests that validate our approach.
 
\section{Linear versus tree representations of numbers\label{ZTree}}

In the standard library of {\sc Coq}, strictly positive numbers are represented as linear structures, low bits first.

\begin{verbatim}
Inductive positive : Set :=
    xI : positive -> positive 
  | xO : positive -> positive 
  | xH : positive.
\end{verbatim}
The constructor {\tt xI} indicates a one digit, the constructor {\tt xO}
a zero digit and the constructor {\tt xH} the final one digit.
For example, 17 and 18 are represented as {\tt xI (xO (xO (xO (xH))))} and
{\tt xO (xI (xO (xO (xH))))} respectively. The choice of the representation
has some direct impact on the way operations are implemented. To illustrate this on an example,
let us consider the comparison function {\tt Pcmp}. It takes two positive numbers
and returns a comparison value
\begin{verbatim}
Inductive comparison: Set := Eq | Lt | Gt.
\end{verbatim}
As numbers are represented low bits first, to compare two numbers 
one needs to walk down both numbers keeping track of what the current status
of the comparison is. This is what the auxiliary function {\tt Pcompare} does. The main 
function {\tt Pcomp} starts the computation with the initial status being equality.
\begin{verbatim}
Fixpoint Pcompare (x y: positive) (r: comparison): comparison :=
  match x, y with
  |    xH,    xH => r
  |    xH,    _  => Lt
  |    _ ,    xH => Gt
  | xI x', xI y' => Pcompare x' y' r
  | xO x', xO y' => Pcompare x' y' r
  | xI x', xO y' => Pcompare x' y' Gt
  | xO x', xI y' => Pcompare x' y' Lt
  end.
  
Definition Pcmp x y := Pcompare x y Eq.
\end{verbatim}
This is clearly not optimal but it is the best one can do with this representation. 
Changing to a high bits first representation would lead to a more efficient
comparison but would penalize other functions like parity checking. With this linear
datastructure, recursive calls only skip a single bit. 
This is a real limitation. Efficient algorithms for large numbers like Karatsuba 
multiplication~\cite{Karat} use a divide and conquer strategy. They require to be able to split numbers
in parts efficiently. 
This motivates our representation based on a tree-like structure. Given an arbitrary word set {\tt w}, we define
the two-word set {\tt w2 w} as follows 
\begin{verbatim}
Inductive w2 (w: Set): Set :=  
   WW : w -> w -> w2 w.
\end{verbatim}
Arbitrarily we choose that high bits are given first, low bits
second. Now using dependent type, we iterate this process and define
numbers of height {\tt n} that contain $2^\texttt{n}$ words.
\begin{verbatim}
Fixpoint word (w: Set) (n:nat): Set :=
 match n with
 |   O => w
 | S n => w2 (word w n)
 end.
\end{verbatim}
To represent a given number exactly, one has to choose an appropriate height.
For example, taking the usal booleans for basic words, a minimum
height of 2 is necessary to represent the number 13. With this height, 
numbers have type {\tt (word bool 2)} and 13 is represented 
as {\tt (WW (WW true false) (WW true true))}.

Following the methodology proposed in~\cite{GreMa} in a similar setting, 
the operations are not going to be defined on the type {\tt word} directly.
Definitions are staggered instead. What is defined is a functor that 
allows to build a two-word modular arithmetic on top of a single-word one. To rephrase it,
when defining a new function, we just explain how to compute the 
result on two-word values knowing how to compute it on single-word values.
To illustrate this, let us go back to our comparison function. 
To mimic the function {\tt Pcompare}, we first suppose the existence
of the comparison on single words
\begin{verbatim}
Variable w_compare: w -> w -> comparison -> comparison.
\end{verbatim}
and then define the function for two-word values
\begin{verbatim}
Definition ww_compare (x y: w2 w) (r: comparison) :=
  match x, y with
     WW xH xL, WW yH yL => w_compare xH yH (w_compare xL yL r) 
  end.
\end{verbatim}
This is not the function that is in our library. We can take 
advantage of the tree-like structure and compare high bits first.  
\begin{verbatim}
Variable w_cmp: w -> w -> comparison.left

Definition ww_cmp (x y: w2 w) :=
  match x, y with
     WW xH xL, WW yH yL => 
       match w_cmp xH yH with 
          Eq  => w_cmp xL yL
       |  cmp => cmp
       end
  end. 
\end{verbatim}
The capacity of choosing between high and low bits alone is
not sufficient to justify our choice of representation. What
is important with this representation is that we get for free
the possibility to split numbers in two. The next section 
explains why this property is crucial to implement efficient
algorithms for functions like multiplication, division, square root.
Note that in term of memory allocation, having a tree structure
does not imply any overhead. Building a tree structure or 
building the equivalent linear list of words require the same number of cells.

One main drawback of our representation is that we manipulate only complete binary trees. 
So even if we choose carefully the appropriate height, half of the words could be unnecessary 
to compute the final result. 
To soften this problem, we have extended the definition of {\tt w2} to include an
empty word {\tt W0}. 
\begin{verbatim}
Inductive w2 (w: Set): Set :=  
   W0: w2
|  WW: w -> w -> w2.
\end{verbatim}
For example, the number 13 can be represented at height 3 as
\begin{verbatim}
 WW W0 (WW (WW true false) (WW true true))
\end{verbatim}
With this extension, we loose the unicity of representation. Still, there is a notion
of canonicity, {\tt W0} should always be preferred to a sub-tree full of zeros. Note that in
our development all functions have been carefully written in order to preserve canonicity but
canonicity  is not part of their specificationlabels. since it is not necessary for ensuring safe computations.
The final version of the comparison function is then
\begin{verbatim}
Definition ww_cmp (x y: w2 w) :=
  match x, y with
  | W0, W0 => Eq
  | W0, WW yH yL =>
    match w_cmp w_0 yH with
    | Eq => w_cmp w_0 yL
    | _  => Lt
    end
  | WW xh xl, W0 =>
    match w_cmp xH w_0 with
    | Eq => w_cmp xL w_0
    | _  => Gt
    end
  | WW xH xL, WW yH yL =>
    match w_cmp xH yH with
    | Eq  => w_cmp xL yL
    | cmp => cmp
    end
  end.
\end{verbatim}
where {\tt w\_0} represents the zero for single word.


\section{The certified library \label{Op}}
The functions that are defined in the library are the following:
addition, subtraction, multiplication,  division, square root, gcd, power function
and modulo. 
For each of them, we not only give an implementation but also formally prove 
that this implementation meets its specification. Specifications are expressed 
using predicates over integers. For this, we use two transfer functions.
Given a single word element {\tt x}, its corresponding value as an integer
is denoted as {\tt [|x|]}. Given a two-word element {\tt y}, its corresponding
value as an integer is denoted as {\tt [[y]]}. The maximum value that can be
expressed with one word plus one is {\tt wB}. From these definitions, the 
following statement holds
\begin{verbatim}
forall x y, [[WW x y]] = [|x|] * wB + [|y|].
\end{verbatim}
Using these definitions, it is possible to state the specifications
of the functions of the library. For example, proving that the comparison
function defined in the previous Section is correct, we have to proved
that if the function {\tt w\_compare} meets its specification
\begin{verbatim}
forall x y,
       match w_compare x y with
       | Eq => [|x|] = [|y|]
       | Lt => [|x|] < [|y|]
       | Gt => [|x|] > [|y|]
       end.
\end{verbatim}
so does the function {\tt ww\_compare}
\begin{verbatim}
forall x y,
       match ww_compare x y with
       | Eq => [[x]] = [[y]]
       | Lt => [[x]] < [[y]]
       | Gt => [[x]] > [[y]]
       end.
\end{verbatim}

\subsection{Words and carries}

For some basic functions like addition and substration, their corresponding
modular implementation if considered as a function that takes two words and
returns a word cannot be exact. The result fits in a word plus a carry.  
To represent carries, the only possibility in a functional language is
to integrate it in the result returned by the function 
\begin{verbatim}
Inductive carry (w: Set): Set :=
  | C0 : w -> carry
  | C1 : w -> carry.
\end{verbatim}
Associated to a carry, there are two interpretation functions.
One interprets positively the carry: {\tt [+|C1 x|] = wB + [|x|]} and
{\tt [+|C0 x|] = [|x|]}. The other one interprets it negatively:
{\tt [-|C1 x|] = [|x|] - wB} and
{\tt [-|C0 x|] = [|x|]}.

To illustrate how carries are manipulated, let us consider the successor function.
In our library it is represented by two functions
\begin{verbatim}
w_sub: w -> w -> w
w_sub_c: w -> w -> carry w
\end{verbatim}
The first function represents the modular version, the second one the exact 
version. With these two functions, it is possible to define the version
for two-words element. For example, the definition of the modular one is
\begin{verbatim}
 Definition ww_succ x :=
  match x with
  | W0 => WW w_0 w_1
  | WW xh xl =>
    match w_succ_c xl with
    | C0 l => WW xh l
    | C1 l => WW (w_succ xh) w_0
    end
  end.
\end{verbatim}
Note that contrary to what happens in imperative languages like C, returning
a carry allocates a memory cell. This has some impact on the implementation:
we try as much as possible to avoid creating them.
For some critical pieces of code we use a continuation passing style,
where functions take two extra arguments that are continuations.
The first continuation corresponds to the code that is called when there is a carry and
the second one when there is not. Another situation that occurs often is when
we know in advance that the result always returns (or similarly always not return a carry).
In that case, we can call directly the modular function. It is  in proving that the
function meets its specification that we will have to justify this knowledge.
An example of such a situation is a naive implementation of the exact function that adds two to a one
word element by calling twice the successor function.
\begin{verbatim}
 Definition w_add2 x :=
  match w_succ_c x with
  | C0 y => w_succ_c y
  | C1 y => C1 (w_succ y)
  end.
\end{verbatim}
In the case where the first increments has created a carry, we are sure that the second
increment cannot raise any carry so we can directly call the {\tt w\_succ} function.

\subsection{Moving bits}

If most of the operations works at word level, some functions like the
shifting operation requires to work at a finer level, the level of bits.
Surprising all the operations we had to perform at bit levels can be built
on top of  one single function
\begin{verbatim}
w_add_mul_div : positive -> w -> w -> w
\end{verbatim}
Evaluating  {\tt (w\_add\_mul\_div p x y)} returns a new word that
is composed for its {\tt p} first bits by the last bits of {\tt x}
and for the remaining bits by the first bits of {\tt y}.
Its specification is the following
\begin{verbatim}
forall x y p, 
 2 ^ p < wB  ->
 [|w_add_mul_div p x y|] =
    ([|x|] * (2 ^ p) + ([|y|] * (2 ^ p)) / wB) mod wB.
\end{verbatim}
Two degenerated versions of this function are of some direct interest. Calling
{\tt w\_add\_mul\_div} with a zero word as second argument implements
the shift left. Calling it with a zero word as third argument implements
the shift right.

\subsection{Divide and conquer algorithms}

\subsubsection{Karatsuba multiplication}
To speed up multiplication was the original motivation of our 
tree representation of numbers. It is represented in our library
by the function
\begin{verbatim}
w_mul: w -> w -> w2 w
\end{verbatim}
and its specification is
\begin{verbatim}
forall x y, [[ w_mul_c x y ]] = [|x|] * [|y|]
\end{verbatim}
The naive implementation on two-word elements follows the simple equation
\begin{verbatim}
[[WW aH aL]] * [[WW bH bL]] =
  [|aH|] * [|bH|] * wB ^ 2 + 
     ([|aH|] * [|bL|] + [|aL|] * [|bH|]) * wB + [|aL|] * [|bL|]
\end{verbatim}
Performing a multiplication requires four submultiplications.
Karatsuba multiplication~\cite{Karat} saves one of these submultiplications
\begin{verbatim}
[[WW aH aL]] * [[WW bH bL]] =
  let v1 := [|aH|] * [|bH|] in
  let v2 := [|aL|] * [|bL|] in
  v1 + (([|aH|] + [|bL|]) * ([|aL|] + [|bH|]) - v1 - v2) * wB + v2
\end{verbatim}
Note that Karatsuba multiplication is more effective than the naive one
only when numbers are getting large enough. This means that our library
includes the two implementations. They are used to define two
different functors. The functor with the naive multiplication to
build operations for trees of small height.

\subsubsection{Recursive Division}
The general division algorithm that we have implemented is the usual schoolboy 
method that iterates the division of two words by one word. 
The key point is how to perform this last division efficiently. 
The algorithm we have implemented is described in~\cite{RecDiv}.
The idea is first to use a recursive call to guess an approximation 
of the quotient and  then to perform an adjustment in order to get the proper quotient.

In our development, the division two by one on single words takes
three words and returns a pair composed of the quotient and the remainder.
\begin{verbatim}
Variable w_div21: w -> w -> w -> w * w
\end{verbatim}
and its specification is 
\begin{verbatim}
forall a1 a2 b,  [|a1|] < [|b|] -> wB / 2 <= [|b|] -> 
  let (q, r) := w_div21 a1 a2 b in
   [|a1|] * wB + [|a2|] = [|q|] * [|b|] + [|r|] /\ 
    0 <= [|r|] < [|b|].
\end{verbatim}
Only the two conditions deserve some explanation.
The first one ensures that the quotient fits in one word.
The second one requires the first bit of the divisor to be a 
one. It ensures that the recursive call computes an approximation 
of the quotient that is not too far from the correct value.

Before defining the function {\tt ww\_div21} for two-word elements,
we need to define the intermediate function  {\tt w\_div32}
that divides three one-word elements by two one-word elements.
Its specification is
\begin{verbatim}
forall a1 a2 a3 b1 b2, 
  [[WW a1 a2]] < [[WW b1 b2]] -> wB / 2 <= [|b1|] ->
  let (q, r) := w_div32 a1 a2 a3 b1 b2 in
  [|a1|] * wB ^ 2 + [|a2|] * wB  + [|a3|] =  
      [|q|] *  ([|b1|] * wB + [|b2|])  + [[r]] /\ 
  0 <= [[r]] < [|b1|] * wB + [|b2|].
\end{verbatim}
The two conditions play the same r�les than the ones for the specification of {\tt w\_div21}.
As the code is a bit intricate, we just explain here how the function
proceeds. It first calls {\tt w\_div21} to divide {\tt a1} and {\tt a2}
by {\tt b1}. This gives a pair {\tt q} and {\tt r} such that
\begin{verbatim}
 [|a1|] * wB + [|a2|] = [|q|] * [|b1|] + [|r|]
\end{verbatim}
{\tt q} is considered as the approximation of quotient.
The second condition in the specification ensures that if this approximation
is not exact, it exceeds the real value of at most two units. So the quotient 
can only be {\tt q}, {\tt q - 1} or {\tt q - 2}. 
As we have 
\begin{verbatim}
 [|a1|] *  wB ^ 2  + [|a2|] * wB + [|a3|] = 
   [|q|] * ([|b1|] * wB  + [|b2|]) + ([|r|] * wB - [|q|] * [|b2|])
\end{verbatim}
we know in which situation we are by testing the sign of the candidate
remainder.
In our modular arithmetic, it amounts in checking if the substraction   {\tt (WW r wO) - (q * b2)} 
produces or not a carry.  
If it is positive (no carry), the quotient is {\tt q}. If it is negative (a carry), we
have to consider {\tt q - 1} and add in consequence {\tt (WW b1 b2)} to the candidate remainder.
We test again the sign of this new candidate. If it is positive, the quotient is {\tt q - 1}
otherwise it is {\tt q - 2}.
Now the definition of {\tt ww\_div21} is straightforward. Not considering the cases for the {\tt W0}
constructor, we get
\begin{verbatim} 
 Definition ww_div21 a1 a2 b :=
  match a1, a2, b with
  ....  
  | WW a1H a1L, WW a2H a2L, WW bH bL =>
        let (q1, r) := w_div32 a1H a1L a2H bH bL in
        match r with
        | W0 => (WW q1 w_0, W0W a2l)
        | WW rH rL =>
          let (q2, s) := w_div32 rH rL a2L b1 b2 in
          (WW q1 q2, s)
        end
  end.
\end{verbatim}
Dividing with {\tt w\_div32} the high part of {\tt a1}, the low part of {\tt a1} and the high part of {\tt a2}
by the high and low parts of {\tt b} gives the high part of the quotient. Dividing again with {\tt w\_div32} 
the high part of the remainder, the low part of the remainder, the low part of {\tt a2} by the high and 
low parts of {\tt b} gives  the low part of the quotient and the remainder.
  
\subsubsection{Recursive Square Root}

The algorithm for computing the square root is similar to the one for division.
It was first described in~\cite{RecSqrt} and has already been formalized in a 
theorem prover~\cite{BerMagZim02}. It requires the number to be splitted
in four. For this reason it is represented by the following function in our
library
\begin{verbatim}
w_sqrt: w -> w -> w * carry w;
\end{verbatim}
it returns the square root and the rest.
Its specification is
\begin{verbatim}
forall x y,
    wB / 4 <= [|x|] ->
    let (s,r) := w_sqrt2 x y in
    [[WW x y]] = [[s]] ^ 2 + [+|r|] /\ [+|r|] <= 2 * [|s|]
\end{verbatim}
As for the division, the input must be sufficient large so that the recursive call
that computes the approximation is accurate enough. 

The definition of the square root needs a support function that implements a division
by two times a number
\begin{verbatim}
w_div2s: carry w -> w -> w -> carry w * carry w
\end{verbatim}
with its specification
\begin{verbatim}
 forall a1 a2 b,
     wB/2 <= [|b|] -> [+|a1|] <= 2 * [|b|] ->
     let (q,r) := w_div2s a1 a2 b in
     [+|a1|] * wB + [|a2|] = [+|q|] *  (2 * [|b|]) + [+|r|] /\ 
     0 <= [+|r|] < 2 * [|b|].
\end{verbatim}
The idea of the algorithm is given by the following equation
\begin{verbatim}
  [[WW aH aL]] * wB ^ 2 + [[WW bH bL]] =
    let (q, r) := w_sqrt aH aL in
    let (q1, r1) := w_div2s q bH q in
    [[WW q q1]] ^ 2 + [+|r1|] * wB + [|bL|] - [|q1|] ^ 2 
\end{verbatim}
It indicates that the element {\tt (WW q q1)} is a possible candidate
for the square root. As a matter of fact, because of the condition on
the input, we are sure that the square root is either {\tt (WW q q1)} or
{\tt (WW q q1) -1}. It is the sign of {\tt [+|r1|] * wB + 
[|bL|] - [|q1|] \^{} 2} that indicates in which situation we are. 
\section{Implementing base word arithmetic \label{word}}

The final step to complete our library is to define the arithmetic for the 
base words. 
Once defined,  we get the modular arithmetic for the desired size by 
applying an appropriate number of times our functors. 
In a classical implementation, these base words would be machine words.
Unfortunately, machine words are not yet accessible from the {\sc Coq} 
language. 

\subsection{Defined modular arithmetic}

For the moment, the only way to have a modular arithmetic for base words 
inside {\sc Coq} is to define base words as a datatype. 
For example, we have for two-bit words
\begin{verbatim}
Inductive word2 : Set := OO | OI | IO  | II.
\end{verbatim}
The functions are then defined by simple case analysis. For example,
the exact successor function is defined as
\begin{verbatim}
Definition word2_succ_c x :=
 match x with
 | OO => C0 OI
 | OI => C0 IO
 | IO => C0 II
 | II => C1 OO
 end.
\end{verbatim}
We also need to give the proofs that every function meets its specification.
These proofs are also done by case analysis. 

Rather than writing by hand functions and proofs, we have
written an {\sc Ocaml} program~\cite{ocaml} instead.
This program takes the word size as argument and generates the 
desired base arithmetics with all its proofs. 
It is a nice application of meta-proving. Unfortunately, 
functions and their corresponding proofs grow quickly with the word size. 
For example, the addition for {\tt word8} is a pattern matching of 
65536 cases. {\tt word8} is actually the largest size {\sc Coq} can handle. 

The main benefit of  this approach is to get an arithmetic library that is 
entirely expressed in the logic of {\sc Coq}. The library is portable: 
no extension of the {\sc Coq} kernel is needed. 

\subsection{Native modular arithmetic}

To test our library with some machine word arithmetic, we use the extraction
mechanism. 
It converts automatically {\sc Coq} functions into {\sc Ocaml} functions.
It is then possible to run the resulting program with the 31-bit native
{\sc Ocaml} arithmetic or a simulated 64 bit arithmetic. 
Not all the functions that we have implemented have their corresponding 
functions in the native modular arithmetic, 
so some native code had to be developed for these functions.
The formal verification of this code has not yet been completed, but
it is interesting to get a idea of the speed-up we can expect if we add 
native arithmetic in Coq.


\section{Evaluating the library \label{bench}}

The complete library with the corresponding contribution for prime numbers is available at \url{http://gforge.inria.fr/projects/coqprime/}. It consits in 9000 lines of hand-written
definitions and proofs. The automatically generated {\tt word8} arithmetic is much bigger,
95 Mb: 41 Mb is used to define functions and 54 Mb for the proofs. This is the largest ever
contribution that has been verified by {\sc Coq}. With this new library, we have been capable to
prove $2^{44497} - 1$ was prime using {\sc Coq}. As far as we know, it is the largest 
prime number  that has been certified by a theorem prover.

Even for small numbers, the {\sc Coq} version of our library is much more efficient than
the standard one. This is illustrated by Figure~\ref{fig:TimeCompW} and by the fifth and sixth  
columns of Figure~\ref{fig:Mersenne}. There is a maximum speed-up of 90. Our library makes it 
really possible to compute with large numbers inside {\sc Coq}.
\begin{figure}
\begin{center}
\begin{tabular}{|l|r| r|r|}
\hline
 & ~digits~ & ~postive~ & ~word8~ \\
\hline
~1234567891       ~  & 10~  & 0.50s~  & 0.10s~  \\
~74747474747474747~  & 17~ & 0.56s~  & 0.12s~  \\
~1111111111111111111~ & 19~ & 0.66s~ & 0.20s~  \\
~$(2^{148}+1)/17$ ~   & 44~ & 2.77s~  & 0.36s~  \\
~$P_{150}$   ~       & 150~ & 75.62s~  & 8.44s~  \\
\hline
\end{tabular}
\end{center}
\caption{Some verifications of certificates with the standard and our {\sc Coq} arithmetics}
\label{fig:TimeCompW}
\end{figure} 
\begin{figure}
\begin{center}
\begin{tabular}{|r|r|r|r|r|r|r|r|r |}
\hline
\# & n & digits & years &  positive & word8 & w31 & w64 & Big\_int\\
\hline
12 &  127 &  39 & 1876 &  0.73s & 0.04s & 0.01s & 0.s & 0.s \\
13 &  521 & 157 & 1952 &  53.00s & 1.85s & 0.02s & 0.02s &  0.s\\
14 &  607 & 183 & 1952 &  84.00s & 2.78s & 0.03s & 0.03s &  0.s\\
15 & 1279 & 386 & 1952 &  827.00s & 20.21s& 0.25s & 0.16s &  0.02s\\
16 & 2203 & 664 & 1952 &  4421.00s & 89.1s & 1.1s & 0.8s &  0.08s\\
17 & 2281 & 687 & 1952 &  4964.00s & 97.59s & 1.21s & 0.82s &  0.09s\\
18 & 3217 & 969 & 1957 &  14680.00s & 237.65s & 2.85s & 2.14s &  0.22s\\
19 & 4253 & 1281 & 1961 &35198.00s & 494.09s& 6.4s & 4.58s &  0.6s\\
20 & 4423 & 1332 & 1961 &  39766.00s & 563.27s & 6.99s & 4.99s &  0.67s\\
21 & 9689 & 2917  & 1963 &   & 5304.08s & 56.1s & 39.98s &  5.89s\\	 
22 & 9941 & 2993  & 1963 &   & 5650.63s & 60.5s & 42.53s &  6.32s\\	 
23 & 11213 & 3376 & 1963 &    & 7607.00s & 80.56s & 57.47s &  11.25s\\ 
24 & 19937 & 6002  & 1971 &  & 34653.12s & 377.24s & 268.09s &  45.75s\\
25 & 21701 & 6533 & 1978 &  &43746.21s & 463.02s & 338.04s &  58.56s \\
26 & 23209 & 6987 & 1979  &  &51210.56s & 538.33s & 403.48s &  88.43s\\
27 & 44497 & 13395 & 1979  &  &282784.09s & 3282.23s & 2208.45s &  476.75s \\
\hline

\end{tabular}
\end{center}
\caption{Time to verify Mersenne numbers}
\label{fig:Mersenne}
\end{figure}

Comparing {\tt word8} with the 31 bit {\sc Ocaml} integer {\tt w31} shows all the benefit we would
get from having machine words in {\sc Coq}. There is a maximum speed-up of 80 with respect to {\tt word8}. This
means a speed-up of 7200 with respect to the standard {\sc Coq} library.

The 64 bit {\sc Ocaml} integer {\tt w64} is a simulated arithmetic (our processor has only 32 bit).
This is why there is not such a gap between {\tt w31} and {\tt w64}. {\sc Big\_int}~\cite{bignum} is the  
standard exact library for {\sc Ocaml}. It has a purely functional interface but 
is written in C. The comparison is not bad. We are only 6 time slower. It is also very interesting
that this gap is getting smaller as numbers get larger. Random tests on addition gives a factor of
4, multiplications a factor of 10.

We are still far away from getting the performance of the {\sc gmp}~\cite{GMP} library. This library is written in
C and uses inplace computation instead. This minimizes considerably the number of memory allocations.
Unfortunately, inplace computation is not compatible with the logic of {\sc Coq}.


\section{Conclusions}

The main contribution of our work is to present a certified library for performing
modular arithmetic. Individual arithmetic functions have already been proved correct,
see for example~\cite{BerMagZim02}. To our knowledge, it is the first time verification
has been applied to a complete library  with non trivial algorithms successfully.
Our motivation was to be able to certify  larger prime numbers. Figures given in Section~\ref{bench} 
prove that this goal has been reached: we are now capable of manipulating numbers with more than 40000 digits.
These tests also show that the library with a native base arithmetic is
reasonably efficient. We hope it will motivate people to integrate machine word arithmetic inside {\sc Coq}.

Expressing the arithmetic in the logic has a prize: no side effect is possible, 
also numbers are allocated progressively not in one block.
A natural continuation of our work would be to prove the correctness of a library with side effect.
This would require a much more intensive verification work since inplace computing
is known to be much harder to verify.
Note that directly integrating an existing library inside the prover with no verification
would go against the philosophy of {\sc Coq} to keep its trusted computing base as small
as possible.

From the methodological point of view, the most interesting aspect of this work
has been the use of the meta-proving technique to generate our base arithmetic. This has proved
to be a very powerful technique. We have used it in an ad-hoc way: files are generated concatenating
strings. Developing a more adequate support for meta-proving inside the prover seems a
very promising future work. Note that meta-proving could also be a solution to get more flexibility
in the proof system. Slightly changing our representation, adding for example to  the {\tt w2} type not only {\tt WO}
but also {\tt W1} and {\tt W-1}, would have a devastating effect on our definitions and proofs.
Meta-proving could be a solution for having a formal development for a family of data-structures rather than
just a single one.

Finally, on December 2005, a new prime Mersenne number has been discovered: $2 ^{30402457} - 1$.
It took 5 days to  perform its Lucas-Lehmer test on a super computer. 
The program uses a very intriguing algorithm to perform 
multiplication~\cite{crandall}. 
Proving the correctness of
such an algorithm seems a very challenging task. 



\section*{Acknowledgement}
We would like to thank one of our anonymous referees for his careful reading and 
specially for his suggestion that simplifies our Karatsuba multiplication.

\bibliographystyle{plain}
\bibliography{main}

\end{document}

