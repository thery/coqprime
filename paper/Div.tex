\subsection{Division}
The algorithm we have implemented is described in~\cite{RecDiv}.
The general division algorithm is the usual schoolboy method that
iterates the division of two words by one word. 
The key point is implementing this last division efficiently. For this, 
the idea is first to use a recursive call to guess an approximation of the quotient and 
then to perform an adjustment in order to get the proper quotient.

In our development, the division two by one on single words takes
three words and returns a pair composed of the quotient and the remainder.
\begin{verbatim}
Variable w_div21: w -> w -> w -> w * w
\end{verbatim}
and its specification is 
\begin{verbatim}
forall a1 a2 b,  [|a1|] < [|b|] -> wB / 2 <= [|b|] -> 
  let (q, r) := w_div21 a1 a2 b in
   [|a1|] * wB + [|a2|] = [|q|] * [|b|] + [|r|] /\ 
    0 <= [|r|] < [|b|].
\end{verbatim}
Only the two conditions deserve some explanation.
The first one ensures that the quotient fits in one word.
The second one requires the first bit of the divisor to be a 
one. It ensures that the recursive call computes an approximation 
of the quotient that is not too far from the correct value.

Before defining the function {\tt ww\_div21} for two-word elements,
we need to define the intermediate function  {\tt w\_div32}
that divides three one-word elements by two one-word elements.
Its specification is
\begin{verbatim}
forall a1 a2 a3 b1 b2, 
  [[WW a1 a2]] < [[WW b1 b2]] -> wB / 2 <= [|b1|] ->
  let (q, r) := w_div32 a1 a2 a3 b1 b2 in
  [|a1|] * wB ^ 2 + [|a2|] * wB  + [|a3|] =  
      [|q|] *  ([|b1|] * wB + [|b2|])  + [[r]] /\ 
  0 <= [[r]] < [|b1|] * wB + [|b2|].
\end{verbatim}
The two conditions play the same r�les than the ones for the specification of {\tt w\_div21}.
As the code is a bit intricate, we just explain here how the function
proceeds. It first calls {\tt w\_div21} to divide {\tt a1} and {\tt a2}
by {\tt b1}. This gives a pair {\tt q} and {\tt r} such that
$$ [|a_1|] \textit{wB} + [|a_2|] = [|q|] * [|b_1|] + [|r|]$$
{\tt q} is considered as the approximation of quotient.
The second condition in the specification ensures that if this approximation
is not exact, it exceeds the real value of at most two units. So the quotient 
can only be {\tt q}, {\tt q - 1} or {\tt q - 2}. 
As we have 
$$ [|a_1|] \textit{wB}^2  + [|a_2|] \textit{wB} + [|a_3|] = [|q|] * ([|b_1|] \textit{wB}  + [|b_2|]) + ([|r|] \textit{wB} - 
[|q|] * [|b_2|]) $$
we know in which situation we are by testing the sign of the candidate
remainder.
In our modular arithmetic, it amounts in checking if the substraction   {\tt (WW r wO) - (q * b2)} 
produces or not a carry.  
If it is positive (no carry), the quotient is {\tt q}. If it is negative (a carry), we
have to consider {\tt q - 1} and add in consequence {\tt (WW b1 b2)} to the candidate remainder.
We test again the sign of this new candidate. If it is positive, the quotient is {\tt q - 1}
otherwise it is {\tt q - 2}.
Now the definition of {\tt ww\_div21} is straightforward. Not considering the cases for the {\tt W0}
constructor, we get
\begin{verbatim} 
 Definition ww_div21 aa1 aa2 bb :=
  match aa1, aa2, bb with
  ....  
  | WW a1H a1L, WW a2H a2L, WW bH bL =>
        let (q1, r) := w_div32 a1H a1L a2H bH bL in
        match r with
        | W0 => (WW q1 w_0, W0W a2l)
        | WW rH rL =>
          let (q2, s) := w_div32 rH rL a2L b1 b2 in
          (WW q1 q2, s)
        end
  end.
\end{verbatim}
Dividing with {\tt w\_div32} the high part of {\tt a1}, the low part of {\tt a1} and the high part of {\tt a2}
by the high and low parts of {\tt b} gives the high part of the quotient. Dividing again with {\tt w\_div32} 
the high part of the remainder, the low part of the remainder, the low part of {\tt a2} by the high and 
low parts of {\tt b} gives  the low part of the quotient and the remainder.
  
