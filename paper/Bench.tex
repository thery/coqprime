\section{Evaluating the library \label{bench}}

In~\cite{GreTheWer} we have presented a way of applying 
the two-level approach for proving primality. It is based on the
notion of prime certificate and more 
precisely of {\it Pocklington certificate}.
A prime certificate is an object that witnesses the primality of a number.
The Pocklington certificates we have been using are justified by the following
theorem given in~\cite{lehmer}:
\begin{theorem}\label{lehmer}
Given a number $n$, a witness $a$ and some pairs 
$(p_1,\alpha_1),\dots,(p_k,\alpha_k)$
 where all the $p_i$ are prime numbers,
 let
 \begin{itemize}
\item[]$F_1 = p_1^{\alpha_1}\dots p_k^{\alpha_k}$
\item[]$R_1 = (n - 1) / F_1$
\item[]$ s = R_1 / (2F_1)$
\item[] $r = R_1 \mod\ (2F_1)$
 \end{itemize}
 it is sufficient for $n$ to be prime that the following conditions hold:
\begin{eqnarray}
F_1 \,\,\hbox{is even},\,\,
R_1 \,\, \hbox{is odd}, \,\,\hbox{and}\,\,
F_1R_1  &=&  n -1\\
(F_1 + 1) (2F_1^2 + (r - 1) F_1 + 1) & >& n\\
a^{n-1} &=& 1 (\mod\ n)\\
\forall i\in\{1,\dots,k\}~\gcd(a^{\frac{n-1}{p_i}}-1,n)&=&1\\
r^2 - 8 s\,\,\mbox{is not a square}\,\,\hbox{or}\,\,s &=& 0
\end{eqnarray}
\end{theorem}
For a prime number $n$, the list 
$[a, p_1, \alpha_1, p_2, \alpha_2, \dots, p_k, \alpha_k]$
represents its Pocklington certificate.
Even if generating a certificate for a given $n$ can be cpu-intensive, 
verifying conditions 1-5 is an order of magnitude simpler. In fact, only
the verification of conditions 1-5 is crucial for asserting the primality. 
It requires safe computation and is done inside {\sc Coq}.
The generation of the certificate is delegated to an external tool.
This is a direct application of the skeptic approach described 
in~\cite{BarBar,HarThe}.
                 
With respect to the standard approach for the same 
problem~\cite{Caprotti_Oostdijk:01pockjsc}, the two-level
approach gives a huge improvement in term of size of the
proof object and in term of time.  
Figure~\ref{fig:TimeComp} illustrates this on some examples 
($P_{150}$ is a prime number of 150 digits and the millenium prime is
a prime number of 2000 digits discovered by John B. Cosgrave).
\begin{figure}
\begin{center}
\begin{tabular}{|l|r|r|r|r|r|r|}
\cline{3-6}
\multicolumn{2}{c}{} & \multicolumn{2}{|c|}{size} &
                                               \multicolumn{2}{c|}{time} \\
\hline
~prime     ~     & \multicolumn{1}{c|}{~digits~ } & ~standard~  &
  \multicolumn{1}{c|}{ ~two-level~ } &
  \multicolumn{1}{c|}{ ~standard~ } &
  \multicolumn{1}{c|}{ ~two-level~ } \\
\hline
~1234567891       ~   & 10~ &  94K~ &  ~0.453K~ & 3.98s~  & 0.50s~  \\
~74747474747474747~   & 17~ & 145K~ &  0.502K~ &   9.87s~ & 0.56s~ \\
~1111111111111111111~ & 19~ & 223K~ &  0.664K~ & 17.41s~  & 0.66s~   \\
~$(2^{148}+1)/17$ ~   & 44~ & 1.2M~ & 0.798K~ & ~350.63s~  & 2.77s~   \\
~$P_{150}$   ~        &150~ &  \_~  & 1.902K~   &  \_~   & 75.62s~  \\
~$\textit{millenium prime}$~   &2000 & \_~   & \_~      & \_~    & \_~ \\
\hline
\end{tabular}
\end{center}
\caption{Some verifications of certificates with the standard and two-level approaches}
\label{fig:TimeComp}
\end{figure} 
Even with the two-level approach, we were not capable to certify large prime numbers 
(> 1000 digits) as the millenium prime.
This was also the case when applying the Lucas-Lehmer test
for proving the primality of Mersenne numbers, 
i.e. numbers that can be written as $2 ^ p -1$.
\begin{theorem}\label{lucas}
Let ($S_n$) be recursively defined by $S_0= 4$ and $S_{n+1} = S_n^2 - 2$,
for $p > 2$, $2^p-1$ is prime if and only if $(2^p -1) | S_{p-2}$.
\end{theorem}
The largest Mersenne number we could certify was $2^{4423} - 1$ that has 1300 digits. 

The idea is then to use our new library for certifying large prime numbers.
The complete library with the corresponding contribution for prime numbers 
is available at \url{http://gforge.inria.fr/projects/coqprime/}. 
It consists in 9000 lines of hand-written definitions and proofs. 
The automatically generated {\tt word8} arithmetic is much bigger,
95 Mb: 41 Mb are used to define functions and 54 Mb for the proofs. 
This is the largest ever contribution that has been verified by {\sc Coq}. 
With this new library, we have been capable to prove that the Mersenne number
$2^{44497} - 1$ was prime using {\sc Coq}. 
As far as we know, it is the largest prime number  that has been certified 
by a theorem prover.

Even for small numbers, the {\sc Coq} version of our library is much more 
efficient than the standard one. 
This is illustrated by Figure~\ref{fig:TimeCompW} and by the fifth and sixth  
columns of Figure~\ref{fig:Mersenne}. 
There is a maximum speed-up of 70. Our library makes it 
really possible to compute with large numbers inside {\sc Coq}.
\begin{figure}
\begin{center}
\begin{tabular}{|l|r| r|r|}
\hline
 & ~digits~ & ~positive~ & ~word8~ \\
\hline
~1234567891       ~  & 10~  & 0.50s~  & 0.10s~  \\
~74747474747474747~  & 17~ & 0.56s~  & 0.12s~  \\
~1111111111111111111~ & 19~ & 0.66s~ & 0.20s~  \\
~$(2^{148}+1)/17$ ~   & 44~ & 2.77s~  & 0.36s~  \\
~$P_{150}$   ~       & 150~ & 75.62s~  & 8.44s~  \\
~$\textit{millenium prime}$~   &2000 & \_~   & 88m~ \\
\hline
\end{tabular}
\end{center}
\caption{Some verifications of certificates with the standard and our {\sc Coq} arithmetics}
\label{fig:TimeCompW}
\end{figure} 
\begin{figure}
\begin{center}
\begin{tabular}{|r|r|r|r|r|r|r|r|r |}
\hline
\# & n & digits & years &  positive & word8 & w31 & w64 & Big\_int\\
\hline
12 &  127 &  39 & 1876 &  0.73s & 0.04s & 0.01s & 0.s & 0.s \\
13 &  521 & 157 & 1952 &  53.00s & 1.85s & 0.02s & 0.02s &  0.s\\
14 &  607 & 183 & 1952 &  84.00s & 2.78s & 0.03s & 0.03s &  0.s\\
15 & 1279 & 386 & 1952 &  827.00s & 20.21s& 0.25s & 0.16s &  0.02s\\
16 & 2203 & 664 & 1952 &  4421.00s & 89.1s & 1.1s & 0.8s &  0.08s\\
17 & 2281 & 687 & 1952 &  4964.00s & 97.59s & 1.21s & 0.82s &  0.09s\\
18 & 3217 & 969 & 1957 &  14680.00s & 237.65s & 2.85s & 2.14s &  0.22s\\
19 & 4253 & 1281 & 1961 &35198.00s & 494.09s& 6.4s & 4.58s &  0.6s\\
20 & 4423 & 1332 & 1961 &  39766.00s & 563.27s & 6.99s & 4.99s &  0.67s\\
21 & 9689 & 2917  & 1963 &   & 5304.08s & 56.1s & 39.98s &  5.89s\\	 
22 & 9941 & 2993  & 1963 &   & 5650.63s & 60.5s & 42.53s &  6.32s\\	 
23 & 11213 & 3376 & 1963 &    & 7607.00s & 80.56s & 57.47s &  11.25s\\ 
24 & 19937 & 6002  & 1971 &  & 34653.12s & 377.24s & 268.09s &  45.75s\\
25 & 21701 & 6533 & 1978 &  &43746.21s & 463.02s & 338.04s &  58.56s \\
26 & 23209 & 6987 & 1979  &  &51210.56s & 538.33s & 403.48s &  88.43s\\
27 & 44497 & 13395 & 1979  &  &282784.09s & 3282.23s & 2208.45s &  476.75s \\
\hline

\end{tabular}
\end{center}
\caption{Time to verify Mersenne numbers}
\label{fig:Mersenne}
\end{figure}

Comparing {\tt word8} with the 31 bit {\sc Ocaml} integer {\tt w31} shows 
all the benefit we would get from having machine words in {\sc Coq}. 
There is a maximum speed-up of 95 with respect to {\tt word8}. This
means a speed-up of 6650 with respect to the standard {\sc Coq} library.

The 64 bit {\sc Ocaml} integer {\tt w64} is a simulated arithmetic 
(our processor has only 32 bits).
This is why there is not such a gap between {\tt w31} and {\tt w64}. 
{\sc Big\_int}~\cite{bignum} is the  standard exact library for {\sc Ocaml}. 
It has a purely functional interface but 
is written in C. The comparison is not bad. We are only 6 time slower. 
It is also very interesting
that this gap is getting smaller as numbers get larger. 
Random tests on addition gives a factor of
4, multiplications a factor of 10.

We are still far away from getting the performance of the {\sc gmp}~\cite{GMP}
 library. This library is written in C and uses inplace computation instead. 
This minimizes considerably the number of memory allocations.
Unfortunately, inplace computation is not compatible with 
the logic of {\sc Coq}.

